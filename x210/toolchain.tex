\chapter{交叉编译工具}

  在基于x86系统的PC上开发其他平台的处理器,通常通过交叉编译来完成.
Linux GCC交叉编译工具主要由以下四部分组成:

\begin{itemize}\itemsep=-3pt
  \item GNU gcc compilers for c, c++,包括编译器、链接器等;
  \item GNU binutil,包括归档、目标程序复制和转换、代码分析调试等工具;
  \item GNU C Library,支持目标代码的 c 语言库;
  \item GNU C header,头文件.
\end{itemize}

\section{准备工作}
  准备一个工作目录和一个安装目录.

	由于Cortex--A8 基于 ARM 体系结构, 所以在基于Cortex--A8 开发过程中
必须使用 ARM 的交叉编译.这个编译器环境将使用下面的 GNU 工具:
针对 ARM 的.最终编译后产生的二进制文件只能在 ARM 架构的处理器上运行.

\subsection{工具链安装}
	GNU 工具链提供完整的源代码,可以在 PC 机上用x86 平台的编译工具编译安装,
也可以直接下载二进制代码包解压安装.

	本实验使用的交叉编译工具链路径是 /usr/local/arm-2009q3, 可执行程序在
 /usr/local/arm-2009q3/bin 目录下. 实验中请将该目录添加到环境变量 ``PATH''
中, 或给出完整的路径和编译程序名, 如
/usr/local/arm-2009q3/bin/arm-none-linux-gnueabi-gcc
