\chapter{MPlayer移植}{移植}

\section{实验目的}
\begin{itemize}
    \item 掌握 Linux 系统中应用软件移植的过程和方法;
    \item 理解软件层次依赖关系。
\end{itemize}

\section{软件介绍}

MPlayer是一款开源的多媒体播放器, 支持几乎所有的音频和视频播放, 以GNU
通用公共许可证发布, 可在各主流操作系统使用, 是 Linux 系统中最重要的播放器
之一。MPlayer 中还包含音视频编码工具 \verb|mencoder|。MPlayer 本身是基于
命令行界面的程序, 不同操作系统、不同发行版可以为它配置不同的图形界面,
使其外观多姿多彩。MPlayer 本身也可以编译成 GUI 方式。

MPlayer 除了可以播放一般的磁盘媒体文件外, 还支持CD、VCD、DVD等多种物理
介质和多种网络媒体 (rtp://、rtsp://、http://、mms:// 等)。视频播放时,
它还支持多种不同格式的外挂字幕。大部分音视频格式
都能通过 FFMPEG 项目(另一个开源项目, 提供音视频编解码库支持) 的
libavcodec 函数库原生支持。对于那些没有开源解码器的格式, MPlayer 使用
二进制的函数库。它能直接调用 Windows 的 DLL (动态链接库)。

\section{编译准备}

下载 MPlayer 源代码 MPlayer-1.0rc2.tar.bz2, libmad-0.15.1b.tar.gz 和
zlib-1.2.8.tar.bz2, 分别将其解压。 libmad 是高品质全定点算法的 MPEG 音频
解码库。

准备一个有操作权限的工作目录, 例如 \textasciitilde/workspace, 作为下面
编译结果的暂存目录。以后的编译过程中, 将编译选项 \verb|--prefix| 设置为
该目录。所有编译完成后再将其内容移至开发板适当位置。缺省的安装路径是
\verb|/usr/local|, 该路径一方面需要 root 权限, 另一方面, 它是主机系统的
一个重要目录, 如果被目标机架构 ARM 的代码覆盖, 可能会影响主机的正常工作。

将交叉编译器路径添加到环境变量 ``PATH'' 中。

\section{编译}
\begin{enumerate}
    \item 进入 libmad-0.15.1b 目录,配置编译环境:

\begin{blockcode}
$ ./configure --enable-fpm=arm --host=arm-linux \
	--disable-debugging --prefix=/home/student/workspace
\end{blockcode}
	选项 \verb|--host| 是编译器前缀。

    \item 编译及安装:  \verb|make install|

        在编译时会提示错误: cc1: error: unrecognized command line option
        ``-fforce-mem''。这是因为 gcc3.4 或更高版本已经将 \verb|fforce-mem|
        选项去除了。只需要在 Makefile 中找到该字符串,将其删除即可。

        编译完成后会将静态库 libmad.a 和动态库 madlib.so 安装到
        /home/student/workspace/lib 目录下。如果是动态链接,
        需要将动态链接库复制到目标系统的 \verb|/usr/lib| 目录下。如果不想用
        动态链接, 可以在上面的编译选项中添加一条 \verb|--disable-shared|。
        此原则同样适用于下面的 zlib 库。

    \item 进入 zlib-1.2.8 解压目录, 按下面的步骤编译安装:

\begin{blockcode}
$ export CC=arm-none-linux-gnueabi-gcc
$ ./configure --prefix=/home/student/workspace --host=arm-linux
$ make && make install
\end{blockcode}

  \item 进入 MPlayer 解压目录, 进行如下配置:
\begin{blockcode}
$ ./configure \
    --cc=arm-linux-gcc \
    --target=arm-linux \
    --enable-static \
    --prefix=/home/student/workspace \
    --disable-mp3lib \
    --disable-dvdread \
    --disable-mencoder \
    --disable-live \
    --enable-mad \
    --disable-armv5te \
    --disable-armv6 \
    --enable-libavcodec_a \
    --enable-ossaudio \
    --extra-cflags='-I/home/student/workspace/include' \
    --extra-ldflags='-L/home/student/workspace/lib'
\end{blockcode}
        上面最后两个选项用到了之前准备的 libmad 和 libz 的头文件及生成的
        库文件路径。配置正确后, 可以用 \verb|make| 命令编译。如果一切正常,
        便可在当前目录下生成可执行文件 \verb|mplayer|。注意最后链接时
        用到的库。如果是动态链接, 这些库需要复制到目标系统的 lib 目录里。
\end{enumerate}

最后, 尝试在目标机上播放一些音视频文件。

\section{扩展功能}
\begin{enumerate}
    \item 尝试编译具有图形用户界面的 MPlayer 播放器
    \item 用 \verb|--enable-mencoder| 选项编译 \verb|mencoder|,
        并利用它进行音视频编码、转码。
\end{enumerate}
