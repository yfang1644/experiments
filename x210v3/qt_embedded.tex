\chapter{Qt/Embedded移植}{QT}

\section{实验目的}
\begin{itemize}
    \item 了解嵌入式GUI---Qt/E 软件开发平台的构架;
    \item 学习 Qt/E 移植的基本步骤与方法。
\end{itemize}

\section{Qt/E 介绍}
Qt/Embedded 是跨平台的 C++ 图形用户界面(GUI)工具包, 它是著名的 Qt 开发商
TrollTech 发布的面向嵌入式系统的 Qt 版本。Qt 是目前 KDE 等项目使用的 GUI
支持库, 许多基于 Qt 的图形界面程序可以非常方便地移植到嵌入式
Qt/Embedded 版本上。自从 Qt/Embedded 发布以来, 有许多嵌入式 Linux 开发商
利用 Qt/Embedded 进行了嵌入式 GUI 的应用开发。

Qt/Embedded 注重于能给用户提供精美的图形界面所需的所有元素, 而且其开发过程
是基于面向对象的编程思想, 并且 Qt/Embedded 支持真正的组件编程。

TrollTech 公司所发布的面向嵌入式系统的 Qt/E 版本提供源代码。用户必须针对
自己的嵌入式硬件平台进行裁剪、编译和移植。尽管 Qt/Embedded 可以裁剪到630KB,
但它对硬件平台具有较高的要求。目前 Qt/Embedded 库主要针对手持式信息终端。

本实验主要完成 Qt/Embedded 在嵌入式实验平台上的移植。
\subsection{Qt/E软件包结构}
Qt/E 系统源码一般包括以下几个软件包:
\begin{itemize}
    \item 触摸屏支持库 tslib.tar.bz2;
    \item Makefile 生成工具 tmake-1.11.tar.gz, 它主要由一些脚本程序组成;
    \item 开发平台编译环境库及工具程序 qt-x11-2.3.2.tar.gz;
    \item 目标平台 Qt/Embedded 核心库。用户可到 TrollTech 主页
        ftp://ftp.trolltech.com/qt/source 或者 https://www.qt.io
        下载Qt/Embedded的某个版本的源代码。本实验推荐版本是
        qt-embedded-2.3.7.tar.gz;
    \item Qt桌面环境 qtopia-free-1.7.0.tar.gz。
\end{itemize}

\section{Qt/E 编译}
\subsection{设置环境}
为了以下编译过程顺利进行, 先将上面的压缩文件全部解压。假设各自被解压到
下面的目录(一般 Qt/E 的移植过程也是按这个顺序进行操作的):
\begin{enumerate}
    \item \textasciitilde/work/tslib
    \item \textasciitilde/work/tmake-1.11
    \item \textasciitilde/work/qt-2.3.2
    \item \textasciitilde/work/qt-embedded-2.3.7
    \item \textasciitilde/work/qtopia-1.7.0
\end{enumerate}

在编译 Qt/Embedded时, 用户在 PC 机上应对编译时所需的环境变量进行设置, 这些
设置的主要参数包括:
\begin{itemize}
    \item \verb|QTDIR| --- Qt解压后的所在的目录
    \item \verb|LD_LIBRARY_PATH| --- Qt共享库存放的目录
    \item \verb|QPEDIR| --- qtopia解压后的所在的目录
    \item \verb|TMAKEPATH| --- tmake编译工具的路径
    \item \verb|TMAKEDIR| --- tmake编译工具的目录
    \item \verb|PATH| --- 包含交叉编译工具 \verb|arm-linux-gcc|的路径
\end{itemize}

根据上面的解压目录, 环境变量应作如下设置:
\begin{blockcode}
$ export QTDIR=~/work/qt-embedded-2.3.7
$ export QPEDIR=~/work/qtopia-1.7.0
$ export LD_LIBRARY_PATH=$QTDIR/lib:$LD_LIBRARY_PATH
$ export TMAKEDIR=~/work/tmake-1.11
$ export TMAKEPATH=~/work/tmake-1.11/lib/qws/linux-arm-g++
$ export PATH=~/work/tmake-1.11/bin:/usr/local/arm-linux/bin:$PATH
\end{blockcode}

\subsection{编译过程}
\subsubsection{编译触摸屏库}
Qt/Embedded 支持鼠标和键盘的操作, 但现在许多嵌入式系统都使用触摸屏作为输入
设备, 所以用户必须将触摸屏的相关操作编译成共享库或静态库。

触摸屏库的编译过程可参考第\ref{ch-ts}章内容。

将编译完成后的库复制到 qt-embedded-2.3.7目录:
\begin{blockcode}
$ cp -a src/.libs/* ../qt-embedded-2.3.7/lib
$ cp -a plugins/.libs/*.so ../qt-embedded-2.3.7/lib
\end{blockcode}

\subsubsection{编译qt-x11工具}
在 \textasciitilde/work/qt-2.3.2 下执行:
\begin{blockcode}
$ export QTDIR=~/work/qt-2.3.2
$ export QTEDIR=~/work/qt-embedded-2.3.7
$ export QPEDIR=~/work/qtopia-free-1.7.0
$ export PATH=$QTDIR/bin:$PATH
$ export LD_LIBRARY_PATH=$QTDIR/lib:$LD_LIBRARY_PATH
$ ./configure -no-opengl -no-xft
$ make
$ make -C tools/qvfb
$ mv tools/qvfb/qvfb bin
$ cp bin/uic $QTEDIR/bin
\end{blockcode}

生成的 \verb|uic| 和 \verb|moc| 作为 Qt 应用程序的转换工具, \verb|qvfb|
是 Qt 开发平台的仿真工具。

\subsubsection{编译 Qt/Embedded}
先将补丁文件里的文件替换掉源码包里的对应文件(主要是针对目标平台触摸屏代码
和编译器的设置), 然后在 \textasciitilde/work/qt-embedded-2.3.7 目录下执行

\begin{blockcode} 
$ export QTDIR=~/work/qt-embedded-2.3.7
$ cp ~/work/qtopia-free-1.7.0/src/qt/qconfig-qpe.h src/tools
$ ./configure -xplatform linux-arm-g++ -qconfig qpe -depths 16 -no-qvfb
$ make sub-src
\end{blockcode}

这一步完成后, 生成目标平台的 Qt 核心库 libqte.so* 等等。

\subsubsection{编译 Qtopia}
在 \textasciitilde/work/qtopia-free-1.7.0/src 下面执行:
\begin{blockcode} 
$ ./configure -platform linux-arm-g++
$ make
\end{blockcode}

编译完成后会产生 apps、bin、doc、etc、help、include、plugins等目录及
目录下的文件。至此, 编译过程基本结束。

\subsection{Qt/Embedded的安装}
准备将待安装的文件放在一个独立的目录下。新建一个目录
\textasciitilde/work/qpe, 将
qtopia-free-1.7.0/src 下面的 apps、bin、etc、plugins、i18n、lib、pics
这些目录连同下面的子目录和文件复制到该目录下, 同时将 qt-embedded-2.3.7/lib
下面的库连同字体目录也复制到 qpe/lib 下(注意保持原来的目录结构)。由于字体
文件比较大, 可适当删除一些不常用的字体库, 保留 *.qpf 文件和 fontdir 文件。
另外, 还要将触摸屏配置文件 \textasciitilde/work/tslib/etc/ts.conf
复制到 etc 目录。

将整个 qpe 目录复制到目标系统文件系统的 /usr 目录下, 再为 qpe 建立一个
启动脚本 (/usr/bin/qpe.sh):
\begin{blockcode} 
$ export QTDIR=/usr/qpe
$ export QPEDIR=/usr/qpe
$ export LANG=zh_CN
$ export LD_LIBRARY_PATH=/usr/qpe/lib:$LD_LIBRARY_PATH
$ export QT_TSLIBDIR=/usr/qpe/lib
$ export TSLIB_CONFFILE=/usr/qpe/etc/ts.conf
$ export TSLIB_PLUGINDIR=/usr/qpe/lib
$ export QWS_MOUSE_PROTO=TPanel:/dev/touchscreen/ucb1x00
$ export KDEDIR=/usr/qpe
$ /usr/bin/ts_calibrate
$ /usr/qpe/bin/qpe &> /dev/null
\end{blockcode}

将Qt/E 系统与 BusyBox 结合, 按第 \ref{ch-fs}
章的方法重新制作文件系统映像; 根据需要, 修改启动脚本 inittab(或rc)的启动
执行步骤, 加入上面的脚本命令。

\subsection{Qt-4.8版本编译}
目前的 Qt 最新版本是 5.9(2017年7月发布)。新版本支持更多的特性、更好的
人机交互体验, 例如支持触摸屏的滑动和多点触控等等。但编译相当耗时。

Qt 以商业版权和 LGPL 版权两种版权协议发布。这里以 Qt-4.8.4 的 opensource
版本为例介绍Qt 的编译移植过程。

与低版本编译顺序类似, 应先编译触摸屏库, 将其安装在指定目录, 并在 Qt
解压目录下面的 mkspecs/qws/linux-arm-g++/qmake.conf 文件中添加如下几行:
\begin{blockcode}
QMAKE_INCDIR    = ~/work/build/include
QMAKE_LIBDIR    = ~/work/build/lib
QMAKE_LIBS      = -lpng -lz -lts
\end{blockcode}

\textasciitilde/work/build 目录为 libpng, libz 和 libts 的交叉编译安装目录。
配置 Qt 编译环境如下命令:
\begin{blockcode}
$ ./configure \
    -prefix ~/work/build \
    -opensource \
    -confirm-license \
    -release -shared \
    -embedded arm \
    -xplatform qws/linux-arm-g++ \
    -depths 16,24,32 \
    -fast \
    -optimized-qmake \
    -no-pch \
    -no-largefile \
    -qt-sql-sqlite \
    -system-zlib -system-libtiff -system-libpng -system-libjpeg \
    -qt-freetype \
    -no-qt3support \
    -no-mmx -no-sse -no-sse2 \
    -no-3dnow \
    -no-openssl \
    -no-opengl \
    -webkit \
    -no-phonon \
    -no-nis \
    -no-cups \
    -no-glib \
    -no-xcursor -no-xfixes -no-xrandr -no-xrender \
    -no-separate-debug-info \
    -make libs -make examples -make tools -make docs \
    -qt-mouse-tslib -qt-mouse-pc -qt-mouse-linuxtp
\end{blockcode}

编译完成后, 将 \textasciitilde/work/build 目录平移到目标文件系统, 仿照低版本的
按照方式设置环境变量。

\section{实验要求}
完成一个 Qt/Embedded 系统的编译和安装。

\newpage
\begin{center} ****************** \end{center}
\tt [附] 编译过程中的一些错误及修正

由于编译器版本及源代码规范性等方面的原因, 对源码软件编译过程中经常会
碰到一些编译错误。对这些编译错误, 最好能根据编译器给出的错误提示, 找到出错
的地方, 有针对性地加以修正。下面是在编译Qt/E 过程中可能出现的一些错误及解决
办法:
\begin{enumerate}
  \item {\bf qt-2.3.2}: include/qvaluestack.h:57: 错误......\\
		修改 include/qvaluestack.h 第 57 行, 将\\
		remove( this-$>$fromLast() ); 改为\\
		this-$>$remove( this-$>$fromLast() );
  \item {\bf qt-embedded-2.3.7}: include/qwindowsystem\_qws.h:229: error:
		'QWSInputMethod' has not been declared\\
		在 include/qwindowsystem\_qws.h 里加上类声明:\\
		class   QWSInputMethod;\\
		class   QWSGestureMethod;
  \item {\bf qt-embedded-2.3.7}: *** [allmoc.o] 错误 1\\
		向前追溯出错位置, 在 include/qsortedlist.h 中, 将第51行改为:\\
		$\sim$QSortedList() \{ this-$>$clear(); \}\\
		同样性质的``错误''还有很多, 取决于编译器版本。这里不再一一列举。
  \item {\bf qtopia-free-1.7.0}: Makefile:10: *** 遗漏分隔符 。 停止。\\
		修改 src/3rdparty/libraries/libavcodec/Makefile, 删除 10、14、18
		行的 --e
  \item {\bf qtopia-free-1.7.0}: libraries/qtopia/backend/event.cpp:404:
		error: ISO C++ says that these are ambiguous,......\\
		C++对操作符``$<$=''理解有歧义。可将局部变量 i 声明为 int。
  \item {\bf qtopia-free-1.7.0}: libraries/qtopia/qdawg.cpp:243: error:
		extra qualification 'QDawgPrivate::......\\
		去掉类定义中的本类声明。
  \item {\bf qtopia-free-1.7.0}: libavformat/img.c:723: error: static
		declaration of 'pgm\_iformat' follows non-static declaration\\
		函数或变量属性声明冲突。
  \item {\bf qtopia-free-1.7.0}: 对'\_\_cxa\_guard\_release'未定义的引用\\
		出现在链接阶段。到编译器路径下找到该函数或变量所属的库(libstdc++.so),
		在编译 qt-embedded-2.3.7 时加上库的链接。
\end{enumerate}
\rm
