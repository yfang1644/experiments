\setcounter{chapter}{0}
\begin{center} \LARGE \bf \textsf{实验报告格式} \end{center}
\large \tt
\begin{enumerate}
  \item 实验目的
  \item 实验内容与要求
  \item 实验设计,包括
  \begin{itemize}
    \item 硬件结构
    \item 软件设计思路
  \end{itemize}
  \item 实验记录与分析(包括关键部分的程序清单、说明,实验测量到的数据、波形,
        观察到的现象及分析等等)
  \item 实验体会、小结及建议
  \item 参考文献
\end{enumerate}
其中前三项应在预习中完成,作为预习报告的一部分。

    实验涉及到的软件,要求画出程序流程,附上关键部分的程序代码及注释,并说明
程序的运行方法和结果。

\begin{center} \char"2655 \char"2656 \end{center}
\iffalse

    *实验提倡独立思考,鼓励创新钻研。教师在指导实验中可适当地配以示教,但不
提倡简单的``有问必答''。
  
    *学生实验结束,须经教师检查签字认可,并及时书写实验报告。实验报告应以原始
记录为依据,可在预习报告的基础上,增添整理实验纪录和分析、实验体会与小结。
实验报告应及时完成。
\fi
\rm\normalsize
