\chapter{实时操作系统RTEMS}{RTOS}

\section{实验目的}
\begin{itemize}\itemsep=-3pt
  \item 了解实时操作系统的特性
\end{itemize}

\section{实时操作系统RTEMS简介}
	许多嵌入式应用对任务的响应时间和处理时间都有非常严格的要求。实时操作系统
(Real-Time Operating Systems)就是基于这样的背景发展起来的,它属于嵌入式操作
系统的一类。其核心特征是实时性,而实时性的本质是任务处理所化费时间的可预测性,
即任务需要在规定的时限内完成。

	遵循GNU公共版权协议的开源操作系统 RTEMS(Real-Time Executive for
Multiprocessor Systems)属于硬实时嵌入式操作系统。该项目最初起于1980年代,
用于军事目的。其中的字母``M''从导弹(Missile)到军事(Military)演化到现在的
``多处理器''的概念。它支持包括 POSIX 在内的多种开放 API 标准,移植了包括
NFS和 FATFS的多种文件系统和 FreeBSD TCP/IP 协议栈。最近的版本4.10.2
\footnote{2015年7月} 支持包括X86、ARM、MIPS、SPARC, POWERPC, TI-C3X等在内的
数十种处理器架构。

\section{编译RTEMS}
\begin{enumerate}
  \item 下载RTEMS源码(https://www.rtems.org)。
  \item 编译工具链
  \item 编译内核
  \begin{enumerate}
    \item 进入 rtems 主目录,执行 ./bootstrap
	\item 进入 c/src/lib/libbsp/arm/beagle,修改与 BeagleBone Black 相关的代码
	\item 在 rtems 主目录下建立 build 目录,在 build 目录执行编译配置
\begin{verbatim}
  ../cofigure --target=arm-rtems4.10 \
    --enable-posix
    --enable-itron \
    --enable-networking \
    --enable-cxx \
    --enable-maintainer-mode \
    --enable-rtemsbsp="beagleboneblack" \
    --enable-tests=samples \
    --prefix=build_install
\end{verbatim}
  \end{enumerate}
\end{enumerate}

正确完成以上编译后,会在 build\_install 目录下生成 RTEMS 内核的二进制文件
和一些静态库,在 build 目录下生成可独立运行的二进制文件如 hello.exe、
ticker.exe等等。他们来自 testsuites/samples 目录下的一些小程序源码,缺省的
configure 配置下被编译。

仿照 Linux 内核镜像的生成过程,制作可执行程序的镜像文件

\begin{verbatim}
  arm-rtems4.10-objcopy -O binary --strip-all hello.exe hello.bin
  gzip -9 hello.bin
  mkimage -A arm -O rtems -T standalone -a 0x80000000 -e 0x80000000 \
	  -n RTEMS -d hello.bin.gz hello.img
\end{verbatim}

将 hello.img 复制到 TF 卡的boot 分区,并在 u--boot 的配置文件中指定加载的
文件。

\section{编译内核镜像 RKI(rtems kernel image)}
上面编译出的 .exe 文件是孤立的任务,如无人---机接口,不能直观地看到运行结果。
RKI 可以在 RTEMS 操作系统上创建一个shell,从而实现交互任务。

\begin{enumerate}
  \item 下载 rki (https://github.com/alanc98/rki.git)
  \item 编译

  make ARCH=arm-rtems4.10 BSP=beagleboneblack
    RTEMS\_BSP\_BASE=rtems/build\_install

  其中 rtems/build\_install 是 RTEMS 内核编译后的安装目录(``-{}-prefix''指向
  的目录)。
\end{enumerate}

  正确完成编译后会生成 rki.bin。使用`mkimage''工具制作内核镜像 kernel.img,
将其复制到 TF 卡 boot 分区并用 u--boot 加载。上电、启动,用串口终端进入 shell。

\begin{figure}
\centering\fontsize{10}{10}\selectfont
\begin{boxedminipage}{.96\textwidth}
\begin{verbatim}
reading kernel.img                                                              
367540 bytes read in 37 ms (9.5 MiB/s)                                          
## Booting kernel from Legacy Image at 80800000 ...                             
   Image Name:   RTEMS                                                          
   Image Type:   ARM RTEMS Kernel Image (gzip compressed)                       
   Data Size:    367476 Bytes = 358.9 KiB                                       
   Load Address: 80000000                                                       
   Entry Point:  80000000                                                       
   Verifying Checksum ... OK                                                    
   Uncompressing Kernel Image ... OK                                            
## Transferring control to RTEMS (at address 80000000) ...                      

RTEMS Beagleboard: am335x-based                                                 

RTEMS Kernel Image Booting                                                      

*** RTEMS Info ***                                                              
COPYRIGHT (c) 1989-2008.                                                        
On-Line Applications Research Corporation (OAR).                                
rtems-4.10.99.0(ARM/ARMv4/beagleboneblack)                                      

BSP Ticks Per Second = 100                                                     
*** End RTEMS info ***                                                          

Populating Root file system from TAR file.                                      
Setting up filesystems.                                                         
Initializing Local Commands.                                                    
Running /shell-init.                                                            
1: mkdir ram                                                                    
2: mkrfs /dev/ramdisk                                                           
3: mount -t rfs /dev/ramdisk /ram                                               
mounted /dev/ramdisk -> /ram                                                    
4: hello                                                                        
Hello RTEMS!                                                                    
Create your own command here!                                                   
Starting shell....                                                              

RTEMS Shell on /dev/console. Use 'help' to list commands.                       
[/] # 
\end{verbatim}
\end{boxedminipage}
\caption{RTEMS 终端}
\end{figure}


\section{实验报告要求}
\begin{itemize}\itemsep=-3pt
  \item 总结 RTEMS 移植方法
  \item 尝试在 RTEMS 系统中编写一个多任务程序
\end{itemize}
