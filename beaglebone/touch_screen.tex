\chapter{触摸屏移植}{触屏}\label{ch-ts}

\section{实验目的}
\begin{itemize}
    \item 了解嵌入式系统中触摸屏的原理;
    \item 学习开源软件的移植方法。
\end{itemize}

\section{Linux系统的触摸屏支持}
触摸屏是目前最简单、方便、自然的一种人机交互方式,在嵌入式系统中得到了
普遍的应用.触摸屏库除了用于支持图形接口环境以外,它本身也可以作为触摸屏应用
软件编程的学习范例.

\subsection{触摸屏的基本原理}
触摸屏是由触摸板和显示屏两部分有机结合在一起构成的设备. 根据不同的感应方式,
触摸板又有电阻式、电容式、声表面波式等不同构成. 早期电阻式触摸板由两层透明的
金属氧化物导电层构成. 当触摸屏被按压时, 平常相互绝缘的两层导电层就在触摸点
位置形成接触. 由于当触摸板在 X 和 Y 方向分布了均匀电场, 按压点相当于在 X 和
Y 方向形成了电阻分压. A/D 转换器对该电压采样便可得到按压点的位置坐标.

电容触摸屏利用人体电流感应进行工作. 当被触碰时, 人体电容和触摸板形成耦合,
触摸屏四个角上的电感应设备可以检测到电流的变化. 控制器通过对这四个电流比例
的计算得到触摸点的位置.

以上提到的几种触摸板, 无论采用何种传感原理, 最终都要通过A/D转换器变成
数字量进行分析计算. Linux 系统内核中完成A/D转换部分, 而触摸屏库则给应用程序
提供方便的接口.

\subsection{内核配置}
Linux操作系统内核支持多种触摸屏设备。在Linux内核源码配置界面中, 找到并
选中正确的驱动, 将其编入内核。

内核中, 触摸屏可以是独立驱动, 也可以加入Event interface。后者通过
/dev/input/eventX 设备存取输入设备的事件。建议在内核配置中也选中
Event interface。

\subsection{触摸屏库 tslib}
下载触摸屏库 tslib-1.0.0.tar.bz2, 并将其解压到工作目录。

进入解压目录, 依次执行下面的操作:
\begin{blockcode}
$ ./autogen.sh
$ ./configure --host=arm-linux --prefix=/path/to/install
$ make
$ make install
\end{blockcode}

以上过程注意事项:
\begin{itemize}
    \item 必须事先设置好环境变量PATH, 加入交叉编译器路径, 否则在``make''中交叉
        编译命令不能正确执行;
    \item \verb|--prefix| 选项用于 \verb|make install| 的安装目录, 请使用
        一个拥有写权限的目录, 编译完成后会将结果集中存放于此。如果不指定
        安装目录, 默认的安装目录一般是 \verb|/usr/local|, 普通用户没有
        写权限, 此时不宜用 \verb|make install| 命令。\verb|make| 命令正确
        完成后, 结果分散在 \verb|src/.libs| (库)和 \verb|plugins/.libs|
        (插件)中, 可通过手工复制到目标系统。

    \item 编译过程中可能会有错误提示: undefined reference to `rpl\_malloc',
        可在 \verb|configure| 之前设置变量
        ``export ac\_cv\_func\_malloc\_0\_nonnull=yes''。
\end{itemize}

正确编译后, 会在安装目录下新生成 \verb|etc|、\verb|bin|、\verb|lib|、
\verb|include| 四个子目录。 \verb|etc|里的 \verb|ts.conf| 是库的配置文件,
\verb|bin| 下面的可执行程序包括触摸屏校准和测试工具, \verb|lib| 里是
触摸屏的动态链接库和模块插件, \verb|include| 下面的 \verb|tslib.h| 可用于
基于触摸屏库的应用程序二次开发。

\subsection{触摸屏库的安装和测试}
将前面产生的文件按目录对应关系分别复制到目标系统的 \verb|/usr| 目录中,
编辑 \verb|ts.conf| 文件, 去掉``\# module\_raw input''前面的注释。
按下面的方式设置环境变量:
\begin{itemize}
    \item \verb|export TSLIB_TSDEVICE=/dev/input/event2|

        触摸屏设备文件或 Event interface 设备文件。eventX的主设备号是13,
        次设备号从 64 开始, 可通过 /proc/bus/input/devices 文件获知
        触摸屏的次设备号。
    \item \verb|export TSLIB_CONFFILE=/etc/ts.conf|

        触摸屏库的配置文件。一般需要保留这几个模块:
    \begin{itemize}
        \item \verb|variance|, 用于过滤AD转换器的随机噪声。它假定触点
            不可能移动太快, 其阈值 (距离的平方)由参数 \verb|delta| 指定;
        \item \verb|dejitter|, 去除抖动, 保持触点坐标平滑变化;
        \item \verb|linear|, 线性坐标变换。
    \end{itemize}
    \item \verb|export TSLIB_PLUGINDIR=/usr/lib/ts|

        插件模块文件 (.so)所在目录。
    \item \verb|export TSLIB_CONSOLEDEVICE=none|

        终端设备, 缺省的是 \verb|/dev/tty|, 此处不需要。
    \item \verb|export TSLIB_FBDEVICE=/dev/fb0|

        帧缓冲设备文件。
    \item \verb|export TSLIB_CALIBFILE=/etc/pointercal|

        校准文件。早期触摸屏由于工艺原因, 每台机器的坐标读取数值差异较大,
        使用前必须通过校准工具将触摸屏和液晶屏坐标进行校准,
        产生一个校准文件。
\end{itemize}

以上准备工作就绪后, 尝试执行 \verb|/usr/bin/ts_test|。

\section{实验内容}
完成触摸屏移植。

分析 ts\_test.c, 利用触摸屏库编写一个能进行触摸屏操作的应用程序, 功能自定。
