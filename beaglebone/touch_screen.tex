\chapter{触摸屏移植}{触屏}\label{ch-ts}

\section{实验目的}
\begin{itemize}\itemsep=-3pt
  \item 了解嵌入式系统中触摸屏的原理;
  \item 学习开源软件的移植方法。
\end{itemize}

\section{Linux系统的触摸屏支持}
	触摸屏是目前最简单、方便、自然的一种人机交互方式,在嵌入式系统中得到了
普遍的应用.触摸屏库除了用于支持图形接口环境以外,它本身也可以作为触摸屏应用
软件编程的学习范例.

\subsection{触摸屏的基本原理}
触摸屏是由触摸板和显示屏两部分有机结合在一起构成的设备. 根据不同的感应方式,
触摸板又有电阻式、电容式、声表面波式等不同构成. 早期电阻式触摸板由两层透明的
金属氧化物导电层构成. 当触摸屏被按压时,平常相互绝缘的两层导电层就在触摸点
位置形成接触. 由于当触摸板在 X 和 Y 方向分布了均匀电场, 按压点相当于在 X 和
Y 方向形成了电阻分压. A/D 转换器对该电压采样便可得到按压点的位置坐标.

电容触摸屏利用人体电流感应进行工作. 当被触碰时, 人体电容和触摸板形成耦合,
触摸屏四个角上的电感应设备可以检测到电流的变化. 控制器通过对这四个电流比例
的计算得到触摸点的位置.

以上提到的几种触摸板, 无论采用何种传感原理, 最终都要通过A/D转换器变成
数字量进行分析计算. Linux 系统内核中完成A/D转换部分, 而触摸屏库则给应用程序
提供方便的接口.

\subsection{内核配置}
	Linux操作系统内核支持多种触摸屏设备。在Linux内核源码配置界面中,找到并
选中正确的驱动,将其编入内核。

	内核中,触摸屏可以是独立驱动,也可以加入Event interface。后者通过
 /dev/input/eventX 设备存取输入设备的事件。建议在内核配置中也选中
 Event interface。

\subsection{触摸屏库 tslib}
	下载触摸屏库 tslib-1.0.0.tar.bz2,并将其解压到工作目录。

	进入解压目录,依次执行下面的操作:

\begin{boxedminipage}{.9\textwidth}
\begin{verbatim}
$ export CC=arm-none-linux-gnueabi-gcc
$ ./autogen.sh
$ ./configure --host=arm-linux --prefix=/path/to/install
$ make install
\end{verbatim}
\end{boxedminipage}

以上过程注意事项:
\begin{itemize}
  \item 必须事先设置好环境变量PATH,加入交叉编译器路径,否则在``make''中交叉
	编译命令不能正确执行;
  \item -{}-prefix 选项用于 ``make install''的安装目录,请使用一个拥有写权限
	的目录,编译完成后会将结果集中存放于此。如果不指定安装目录,默认的安装
	目录一般是 /usr/local,普通用户没有写权限,此时不宜用``make install''命令,
	可以仅用``make''命令,结果分散在 src/.libs(库)和 plugins/.libs(插件)中。
  \item 编译过程中可能会有错误提示 ``undefined reference to `rpl\_malloc' '',
	可在 configure 之前设置环境变量``export ac\_cv\_func\_malloc\_0\_nonnull
	=yes''。
\end{itemize}

	正确编译后,会在安装目录下新生成 etc、bin、lib、include 四个子目录。
etc里的 ts.conf 是库的配置文件,bin下面的可执行程序包括触摸屏校准和测试工具,
lib里是触摸屏的动态链接库和模块插件,include 下面的 tslib.h 可用于基于触摸屏库
的应用程序二次开发。

\subsection{触摸屏库的安装和测试}
	将前面产生的文件按目录对应关系分别复制到目标系统的 /usr 目录中,编辑 ts.conf
文件,去掉``\# module\_raw input''前面的注释。按下面的方式设置环境变量:

\begin{itemize}
  \item export TSLIB\_TSDEVICE=/dev/input/event2

	触摸屏设备文件或Event interface设备文件。eventX的主设备号是13,次设备号
从64开始,可通过 /proc/bus/input/devices 文件获知触摸屏的次设备号。
  \item export TSLIB\_CONFFILE=/etc/ts.conf         

	触摸屏库的配置文件。一般需要保留这几个模块:
  \begin{itemize}
    \item variance, 用于过滤AD转换器的随机噪声。它假定触点不可能移动太快,
	其阈值(距离的平方)由参数 delta 指定。
    \item dejitter, 去除抖动,保持触点坐标平滑变化。
    \item linear, 线性坐标变换。
  \end{itemize}
  \item export TSLIB\_PLUGINDIR=/usr/lib/ts

	插件模块文件(.so)
  \item export TSLIB\_CONSOLEDEVICE=none

	终端设备,缺省的是 /dev/tty,此处不需要。
  \item export TSLIB\_FBDEVICE=/dev/fb0

	帧缓冲设备文件
  \item export TSLIB\_CALIBFILE=/etc/pointercal

	校准文件。早期触摸屏由于工艺原因,每台机器的坐标读取数值差异较大,使用前
必须通过校准工具将触摸屏和液晶屏坐标进行校准,产生一个校准文件。
\end{itemize}

	以上准备工作就绪后,尝试执行 /usr/bin/ts\_test.

\section{实验内容}
	完成触摸屏移植。

	分析 ts\_test.c,利用触摸屏库编写一个能进行触摸屏操作的应用程序,功能自定。
