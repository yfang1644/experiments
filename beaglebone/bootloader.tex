\chapter{BootLoader}{Boot}

\section{实验目的}
\begin{itemize}
    \item 了解操作系统的启动过程
    \item 学习制作可引导的存储卡
\end{itemize}

\section{实验步骤}
\subsection{配置bootloader选项}
在下载的 u--boot 目录中, 执行 \verb|make am335x_evm_defconfig|, 即完成
针对 BB--Black 的缺省 bootloader 配置。也可以用 \verb|make menuconfig|
进入菜单界面, 对bootloader 的单个选项进行合理的取舍。

完成配置后, 用``make ARCH=arm CROSS\_COMPILE=arm-none-linux-gnueabi-''
进行编译。``arm-none-linux-gnueabi-''是 ARM 编译器的前缀。编译完成后,
生成 MLO 和 u--boot.img。

\subsection{制作TF卡}
将前面生成的 MLO 和 u--boot.img 用下面的命令写入 TF 卡, BeagleBone Black
上电时按住S2, 系统首先选择MMC0(microSD卡插槽)启动。

\begin{blockcode}
# dd if=MLO of=/dev/mmcblk0 count=1 seek=1 bs=128k
# dd if=u-boot.img of=/dev/mmcblk0 count=2 seek=1 bs=384k
\end{blockcode}

再在该卡上创建两个文件系统, 一个用于存放启动内核相关的固件及内核映像, 用
VFAT格式化; 一个用于准备作为根文件系统, 用 inode 型文件系统格式化(Ext4FS、
BtrFS、ReiserFS 等等均可, 要求内核支持)。前者不宜过大, 够用即可。

\begin{blockcode}
# mkfs.vfat /dev/mmcblk0p1
# mkfs.ext4 /dev/mmcblk0p2
\end{blockcode}

设备名 /dev/mmcblkX 因时而异。如果通过 USB 转接, 设备名也可能是 /dev/sdbX。
请认清操作对象, 错误的操作可能会破坏重要数据。
此外注意在分区时不要覆盖之前写入 TF 卡的裸数据所在扇区。

如果要求系统能自动引导操作系统, 还需要在引导分区上建立u--boot的脚本文件。
u--boot 默认的脚本文件名是 uEnv.txt。它至少应包含下面的内容:
\begin{itemize}
    \item 将指定的内核映像文件加载到内存的指定位置
    \item 设定向内核传递的启动参数, 特别重要的是根文件系统的类型和位置
    \item 跳转到内核解压的内存地址开始运行, 向内核交权
\end{itemize}

一个典型的 uEnv.txt 内容如下:
\begin{blockcode}
loadkernel=fatload mmc 0 0x82000000 zImage
loadfdt=fatload mmc 0 0x88000000 /dtbs/am335x-boneblack.dtb

rootfs=root=/dev/mmcblk0p2 rw rootfstype=ext4
loadfiles=run loadkernel; run loadfdt
mmcargs=setenv bootargs console=ttyO0,115200n8 ${rootfs}

uenvcmd=run loadfiles; run mmcargs; bootz 0x82000000 - 0x88000000
\end{blockcode}
