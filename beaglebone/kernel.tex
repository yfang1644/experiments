\chapter{Linux内核配置和编译}{内核}

\section{实验目的}
\begin{itemize}\itemsep=-3pt
  \item 了解 Linux 内核源代码的目录结构及各目录的相关内容
  \item 了解 Linux 内核各配置选项内容和作用
  \item 掌握 Linux 内核的编译过程
\end{itemize}

\section{相关知识}
\subsection{内核源代码目录介绍}
	Linux 内核源代码可以从网上下载(http://www.kernel.org/pub/linux)。
一般主机平台的Linux内核源码在 /usr/src/linux 目录下。内核源代码的文件按树形
结构进行组织,在源代码树的最上层可以看到如下一些目录:
\begin{itemize}\itemsep=-3pt
  \item arch:arch 子目录包括所有与体系结构相关的内核代码。arch 的每一个
		子目录都代表一个 Linux 所支持的体系结构。例如 arm 目录下就是 ARM
		体系架构的处理器目录,例如Samsung的 S3C系列 mach--s3cXXXX、TI的
		davinci 系列 mach--davinci以及 OMAP系列 mach--omapX。
  \item include:include 子目录包括编译内核所需要的头文件。与 ARM 相关的
		头文件在 include/asm-arm 子目录下。
  \item init:这个目录包含内核的初始化代码(不是系统的引导代码),其中所包含
		main.c 和version.c 文件是研究 Linux 内核的起点。
  \item mm:该目录包含所有独立于 CPU 体系结构的内存管理代码,如页式存储管理、
		内存的分配和释放等。与 ARM 体系结构相关的代码在 arch/arm/mm 中。
  \item kernel:这里包括主要的内核代码。此目录下的文件实现大多数 Linux 的
		内核函数。
  \item drivers:此目录存放系统所有的设备驱动程序,每种驱动程序各占一个子目录:
  \begin{enumerate}\itemsep=-3pt
  \item block:块设备驱动程序。块设备包括 IDE 和 scsi 设备;
  \item char:字符设备驱动程序。如串口、鼠标等;
  \item cdrom:包含 Linux 所有的 CD-ROM 代码;
  \item pci:PCI 卡驱动程序代码,包含 PCI 子系统映射和初始化代码等;
  \item scsi:包含所有的 SCSI 代码以及 Linux 所支持的所有的 SCSI 设备驱动
		程序代码;
  \item net:网络设备驱动程序;
  \item sound:声卡设备驱动程序;
  \end{enumerate}
  \item lib:放置内核的库代码;
  \item net:包含内核与网络的相关的代码;
  \item ipc:包含内核进程通信的代码;
  \item fs: fs 目录是所有的文件系统代码和各种类型的文件操作代码,它的每一个
		子目录支持一个文件系统,如 jffs2、fat、ext2等等。
  \item scripts:目录包含用于配置内核的脚本文件等。每个目录下一般都有depend
		文件和一个 Makefile 文件,他们是编译时使用的辅助文件。仔细阅读这
		两个文件对弄清各个文件之间的相互依托关系很有帮助。
\end{itemize}

\subsection{内核的配置的基本结构}
	Linux 内核的配置系统由四个部分组成:
\begin{enumerate}\itemsep=-3pt
  \item Makefile:分布在 Linux 内核源码中的 Makefile,定义 Linux 内核的
		编译规则。顶层 Makefile 是整个内核配置、编译的总体控制文件;
  \item 配置文件 config.in:给用户提供配置选择的功能。``.config''是内核配置
		文件,包括由用户选择的配置选项,用来存放内核配置后的结果;
  \item 配置工具:包括对配置脚本中使用的配置命令进行解释的配置命令解释器和
		配置用户界面(基于字符界面: make config,基于 ncurses 界面:make 
		menuconfig,基于 X-Window 图形界面:make xconfig);
  \item Rules.make:规则文件,被所有的 Makefile 使用。
\end{enumerate}
\subsection{编译规则 Makefile}
	利用 make menuconfig(或 make config、make xconfig)对 Linux 内核进行
配置后,系统将产生配置文件``.config''。之前的配置文件备份到``.config.old'',
以便用 make oldconfig 恢复上一次的配置。

	编译时,顶层 Makefile 完成产生核心文件(vmlinux )和内核模块(module)两个
任务,为了达到此目的,顶层 Makefile 将读取.config中的配置选项,递归进入到
内核的各个子目录中,分别调用位于这些子目录中的 Makefile 进行编译。
配置文件(.config)中有许多配置变量设置,用来说明用户配置的结果。例如
``CONFIG\_ MODULES=y''表明用户选择了Linux 内核的模块功能。

	配置文件(.config)被顶层 Makefile 包含后,就形成许多的配置变量,每个配置
变量具有四种不同的取值:
\begin{itemize}\itemsep=-3pt
  \item y --- 表示本编译选项对应的内核代码被静态编译进 Linux 内核;
  \item m --- 表示本编译选项对应的内核代码被编译成模块;
  \item n --- 表示不选择此编译选项;
  \item 如果根本就没有选择,那么配置变量的值为空。
\end{itemize}

	除了 Makefile 的编写外,另外一个重要的工作就是把新增功能加入到 Linux 的
配置选项中来提供功能的说明,让用户有机会选择新增功能项。Linux 所有选项配置
都需要在 config.in 文件中用配置语言来编写配置脚本,然后顶层 Makefile 调用
scripts/Configure, 按照 arch/arm/config.in 来进行配置。命令执行完后生成
保存有配置信息的配置文件 .config。

\section{编译内核}
\subsection{Makefile 的选项参数}
	编译 Linux 内核常用的 make 命令选项包括 config、dep、clean、
mrproper、zImage、bzImage、modules、modules\_install等等。
\begin{enumerate}\itemsep=-3pt
  \item config/xconfig/menuconfig:内核配置。调用./scripts/Configure 按照
		arch/arm/config.in 来进行配置。命令执行后产生文件``.config'',
		其中保存着配置信息。下次在做 make config 时将产生新的``.config''
		文件,原文件 ``.config'' 更名为 ``.config.old''。
  \item dep:建立依赖关系,产生两个文件``.depend'' 和``.hdepend'',其中
		``.hdepend''表示每个 .h 文件都包含其他哪些嵌入文件,而``.depend''
		文件有多个,在每个会产生目标文件 .o 文件的目录下均存在,它表示每个
		目标文件都依赖于哪些嵌入文件 .h。2.6以上版本的内核编译不需要这个步骤。
  \item clean:清除以前构核所产生的所有的目标文件、模块文件、核心以及一些
		临时文件等,不产生任何新文件。
  \item mrproper:删除以前在构核过程产生的所有文件,即除了做 make clean 外,
		还要删除``.config''、``.depend'' 等文件,把核心源码恢复到最原始的
		状态。下次构核时必须进行重新配置。
  \item make,make zImage, make bzImage,编译内核,通过各目录的 Makefile
		文件进行,会在各个目录下产生一大堆目标文件。如核心代码没有错误,将
		产生文件 vmlinux,这就是所构的核心。同时产生映像文件 System.map。
		zImage 和 bzImage 选项是在 make 的基础上产生 压缩的核心映像文件。
		正常编译完成后, 生成的 ARM 内核 zImage 文件在目录 arch/arm/boot 中。
		需将其移至 tftp 服务器目录供下载。
  \item modules、modules\_install:模块编译和安装。当内核配置中有选择模块时,
		这些代码不被编入内核文件 vmlinuz,而是通过``make modules'' 编译
		成独立的模块文件 .ko (2.4版本的模块扩展名是 .o)。``make
		modules\_install''将这些文件复制到内核模块文件目录(PC机上通常是
		/lib/modules/\$VERSION/......)。
\end{enumerate}

	2.6版本以后的内核编译过程大大简化,只需要执行 make menuconfig$\to$
make, 即可生成目标内核镜像 zImage,同时被选为模块配置的也编译完成,生成
相应的 .ko 文件。make modules\_install 命令即可将他们复制到指定目录。

\subsection{内核配置项介绍}
	内核配置主目录有下面这些分支:
\begin{enumerate}\itemsep=-3pt
  \item General setup, 内核配置选项和编译方式等等
  \item Enable loadable module support, 利用模块化功能可将不常用的设备驱动
		或功能作为模块放在内核外部,必要时动态地加载。操作结束后还可以从
		内存中删除。这样可以有效地使用内存,同时也可减小了内核的大小。

		模块可以自行编译并具有独立的功能。即使需要改变模块的功能,也不用对
		整个内核进行修改。文件系统、设备驱动、二进制格式等很多功能都支持模块。
		开发过程中通常都需要选中这项。
  \item System Type,处理器架构及相关选项, 根据开发的对象选择。
  \item Networking options,网络配置。
  \item Device Drivers, 设备驱动。包括针对硬盘、CDROM 等以 block 为单位进行
		操作的存储装置和以数据流方式进行操作的字符设备, 还包括网络设备、
		USB设备、多媒体接口、图形接口、声卡等等。和系统硬件相关的配置主要在
		这里。
  \item File systems,文件系统。对 Linux 可访问的各个文件系统的设置。所有的
		操作系统都具有固有的文件系统格式。一般为了对不同操作系统的文件系统
		进行读写操作需安装特殊的应用程序。但是在 Linux 系统中可以通过
		kernel 模块完成这些操作。
\end{enumerate}

\section{实验内容}
	配置一个完整的内核,尽可能理解配置选项在操作系统中的作用;

	将编译的内核文件复制到 tftp 服务器目录,在目标机中下载并运行:

\begin{boxedminipage}{.9\textwidth}
\begin{verbatim}
	u-boot # tftp 80800000 zImage
	u-boot # bootm 80800000
\end{verbatim}
\end{boxedminipage}

\section{实验报告要求}
\begin{itemize}\itemsep=-3pt
  \item 总结内核映像文件的生成方法及对操作系统的作用;
  \item 内核配置中,哪些选项对操作系统的正常启动是必须的?
\end{itemize}

