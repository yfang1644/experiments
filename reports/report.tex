\documentclass[nofonts]{ctexart}
%\usepackage[colorlinks, linkcolor=black,CJKbookmarks]{hyperref}
\usepackage[colorlinks,linkcolor=black]{hyperref}
\usepackage{graphicx}
\usepackage[left=2.5cm,right=2.5cm,top=2.5cm,bottom=2.5cm]{geometry}
\usepackage[utf8]{inputenc}
\usepackage{amssymb}
\usepackage{indentfirst}
\usepackage{tikz}

\newcommand\figureformat{}
\newcommand\tableformat{}
\newcommand\partformat{}

\setCJKfamilyfont{songti}{NSimSun}
\setCJKfamilyfont{fangsong}{STFangsong}
\setCJKfamilyfont{heiti}{WenQuanYi Zen Hei}
\setCJKfamilyfont{kaiti}{AR PL UKai CN}
\setCJKfamilyfont{xinwei}{STXinwei}
\setCJKfamilyfont{lishu}{LiSu}
\setCJKmainfont[BoldFont={WenQuanYi Zen Hei},ItalicFont={STFangsong}]{NSimSun}
\setCJKsansfont{WenQuanYi Zen Hei}
\setCJKmonofont{AR PL UKai CN}
\newcommand*{\song}{\CJKfamily{songti}}
\newcommand*{\fs}{\CJKfamily{fangsong}}
\newcommand*{\hei}{\CJKfamily{heiti}}  
\newcommand*{\kai}{\CJKfamily{kaiti}}  
\newcommand*{\wei}{\CJKfamily{xinwei}} 
\newcommand*{\lishu}{\CJKfamily{lishu}}
\newcommand*{\cy}{\CJKfamily{caiyun}}  

\CTEXsetup[titleformat=\hei,format={\raggedright \zihao{3}}]{section}

\begin{document}

\title{
\lishu \LARGE 嵌入式系统实验 \\
	实验报告要求 \\\  \\
	--------------------------
}
\date{\today}
\maketitle

\large

\setlength{\parindent}{2em}

\section{基本要求}
  实验报告是对实验者对实验过程的总结和思考。实验报告应建立在对实验原理准确
理解的基础上。报告中反映的实验过程应真实可信,有据可查,
{\color{red}不得伪造和臆测}。引用的数据、结论应注明来源,避免成段地引用他人
文字(包括实验讲义)。如有必要,请用自己的语言重新组织、归纳。
{\color{red}严禁抄袭}。


   由于学校教务部门要求期末上交所有实验报告,而收缴的报告今后不会有任何人再
关注。考虑到实验报告是同学们的心血,为便于学校对教学的规($w\acute{u}$)范
($li\acute{a}o$)管理,
本课程报告一律以电子版 pdf 格式提交,可以通过邮箱、面交、QQ等任何你认为
方便的方式提交。我收到报告后会一一回复确认,批阅后会以同样的方式发还。期末时
由我个人负责归档电子版上交教务,同学们的报告自行保管。

   之所以建议采用 pdf 格式,原因有三:

\begin{enumerate}
  \item pdf 可以确保上交的报告不会被我恶意篡改
  \item pdf 不会因阅读器以及阅读器的版本导致文档变形
  \item pdf 格式是开放的,有多款免费、开源阅读器,支持所有操作系统
\end{enumerate}

   对于没有个人计算机的同学,实验中心机房的文档编辑软件完全可以满足要求。
确有喜欢手书报告的同学,报告批阅意见交流后,原版不退,期末上交教务,请理解。

本学期实验,实验报告的正文内容不要超过A4篇幅的12页。程序清单不算篇幅,可以放进
附录,但流程图计入篇幅。报告不要求完整的程序清单。正文中只需引用你认为有意义
的代码片段。

\section{内容要求}
   实验报告内容一般包括下面几个部分:
\begin{enumerate}
  \item 实验目的

  实验目的应明确、集中。一般每个实验的目的不宜超过三个,用词可以比较笼统。
  \item 实验原理/实验介绍

    概述性文字,不要占用过大篇幅,特别是对工程类实验。
  \item 实验内容/实验过程

    这是实验报告的重点之一。要有完整的实验方案,比较详细的实验过程(但忌流水账
式的文字)。包括实验中碰到的问题,产生问题的原因,解决问题的方法等等。这部分
反映实验者的工作努力。他人通过这部分的叙述应能够重复出你的实验。

  \item 实验结果及讨论

  这部分也是实验报告的重点,体现实验者对实验的思考。

  任何实验都应该有明确的结果。本课程对实验结果的正确性不做要求(也就是说,实验
做得对不对,在我们这门课里没有意义),但报告中要反映出因果关系。对于明显与预期
  不符的结果,建议进行充分的讨论和研究。

  \item 扩展空间、建议、抱怨及其他

  实验做完之后,是否有意犹未尽之感,有什么收获,对后人有哪些建议。。。
  (这部分不是一般科学论文的要求,我们也不做要求。有说就说,忌言不由衷)

  \item 参考资料/参考文献

    引用/参考到的网上资源也请在这里列出,尽管网上资源有时效性。通过这种
	方式表示对他人成果的尊重,这也是科学素养的一部分。
\end{enumerate}

   除第一个实验基本上是重复性实验外,后面的实验应以{\color{red}研究型}、
{\color{red}创新型}为主。这两点是实验报告优等评分的必要条件(说明一下,
实验报告成绩不是最终课程成绩,有些话不好明说,你懂的。。。)。每人可以根据自己
的兴趣和能力从{\color{red}任意点}切入,或者对已学过的知识进行深化,或者
借助实验平台进行不同于常人的尝试,都是值得鼓励的行为。如果不打算在这门课上
太多花太多精力,只要完成基本要求也可以接受,对研究性、创新性不做强制要求。


   附件“嵌入式Qt/Embedded抑制”提供一个实验报告的样板,供参考。我对这篇报告的
评价大概是85分。我自认为实验内容已表述清楚,参考这个报告基本能重复出实验结果
(因为这不是嵌入式实验的第一个实验,所以省去了对实验平台、开发平台及开发方法
 的介绍过)。但报告没有深度,也缺乏创新。

\section{格式和文风}
   本课程对报告的文采没有特别要求,嬉笑怒骂皆成文章,只要能把问题说清楚即可,
但不提倡言语粗俗。此外建议对科学概念,用词应力求准确。

   格式虽没有统一的要求,不计入评分标准,但也请力求规范。

   对于上交的报告,我不会未经报告人同意以任何形式发布在公开场合。但考虑到
这两种情况: 1. 教务部门会保留有同学们上交的电子版; 2. 本课程被列入翻转课堂
计划。请同学们自己权衡相关风险和利益。如有特别要求,可以在报告中加上版权声明、
引用声明或者加上水印,防止滥用。



\end{document}

