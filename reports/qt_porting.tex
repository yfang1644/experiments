\documentclass[nofonts]{ctexart}
%\usepackage[colorlinks, linkcolor=black,CJKbookmarks]{hyperref}
\usepackage[colorlinks,linkcolor=black]{hyperref}
\usepackage{graphicx}
\usepackage[left=2.5cm,right=2.5cm,top=2.5cm,bottom=2.5cm]{geometry}
\usepackage[utf8]{inputenc}
\usepackage{amssymb}
\usepackage{indentfirst}
\usepackage{tikz}

\newcommand\figureformat{}
\newcommand\tableformat{}
\newcommand\partformat{}

\setCJKfamilyfont{songti}{NSimSun}
\setCJKfamilyfont{fangsong}{STFangsong}
\setCJKfamilyfont{heiti}{WenQuanYi Zen Hei}
\setCJKfamilyfont{kaiti}{AR PL UKai CN}
\setCJKfamilyfont{xinwei}{STXinwei}
\setCJKfamilyfont{lishu}{LiSu}
\setCJKmainfont[BoldFont={WenQuanYi Zen Hei},ItalicFont={STFangsong}]{NSimSun}
\setCJKsansfont{WenQuanYi Zen Hei}
\setCJKmonofont{AR PL UKai CN}
\newcommand*{\song}{\CJKfamily{songti}}
\newcommand*{\fs}{\CJKfamily{fangsong}}
\newcommand*{\hei}{\CJKfamily{heiti}}  
\newcommand*{\kai}{\CJKfamily{kaiti}}  
\newcommand*{\wei}{\CJKfamily{xinwei}} 
\newcommand*{\lishu}{\CJKfamily{lishu}}
\newcommand*{\cy}{\CJKfamily{caiyun}}  

\CTEXsetup[titleformat=\hei,format={\raggedright \zihao{3}}]{section}

\DeclareUnicodeCharacter{25A1}{\b}

\begin{document}
\sloppy
\renewcommand\contentsname{目录}
\title{
\begin{flushleft}
\hei \hskip 3cm \small 实验报告\\ \ \\
\end{flushleft}
\lishu \LARGE 在嵌入式系统PXA270上移植 \\
	Qt/Embedded\footnote{本文档成型于2010年10月,亿道嵌入式实验系统平台EEliod}
\\
\rule[10pt]{.6\textwidth}{2pt}
}
\author{方元\\ \ \small 南京大学电子学院}
\date{\today}
\maketitle

\setlength{\parindent}{2em}

\begin{center} \begin{minipage}[t]{.8\textwidth}
\kai\baselineskip=4pt
\setlength\parindent{2em}\Large
\abstract{图形用户接口为嵌入式系统提供友好的人机交互界面。本实验基于PXA270
实验平台,讨论Qt的移植问题。(30--50字)}\\
{\hei 关键词:GUI、QTOPIA、PXA270}
\end{minipage}
\end{center}

{\hei 实验目的:}
\begin{itemize}\itemsep=-2pt
  \item 了解图形用户接口在嵌入式系统中的地位
  \item 掌握Qt/Embedded移植的方法
\end{itemize}\vskip 1cm

\tableofcontents\vskip 1cm

\baselineskip=16pt
\section{概述}
\subsection{什么是GUI}
   $\square \square \square \square $ (以下略去123字)\vskip 5cm
\subsection{Qt介绍}
	Trolltech 公司开源软件。目前最新版本 Qt4.5\footnote{2010年10月},
有专业版和免费版。本实验移植嵌入式Qt-2.3.7和桌面环境 Qtopia-1.7.0。
	$\square\square\square$
(以下略去321字)
\vskip 5cm

\subsection{软件环境}
	本实验基于以下软件:
\begin{itemize}\itemsep=-4pt
  \item tslib-1.3, 触摸屏库和应用软件,\textbf{应用部分主要由触摸屏校准和测试构成。}
  \item tmake-1.11, Qt编译脚本,用于生成 Makefile 文件。本实验暂没用到。
  \item qt-x11-2.3.2, (以下简称Qt2) X11环境的Qt,供主机平台使用的Qt环境,包括
  \begin{enumerate}
	\item designer, 图形用户界面GUI设计环境,主要用于开发基于Qt的应用软件
	\item qvfb, Qt- virtual framebuffer, 主机平台的仿真环境
	\item uic, moc, Qt 代码转换器
  \end{enumerate}
  \item qt-embedded-2.3.7,(以下简称 QTE)嵌入式Qt,供目标平台使用的应用软件包
  \item qtopia-free-1.7.0,(以下简称 QPE) Qt桌面环境
  \item 主机编译器 gcc, version 4.3.2
  \item 交叉编译器 arm-linux-gcc, version 3.3.2
\end{itemize}
\section{软件编译}
\subsection{准备工作}
	下载(获取)源代码。将其解压至个人目录的一个专用目录$\sim$work下。
tar.bz2软件包用 tar jxvf 解压,tar.gz 包用 tar zxvf 解压。解压后分别构成以下
几个目录:\footnote{严格地说,还应包括针对特定硬件平台的补丁}
\begin{itemize}\itemsep=-4pt
  \item tslib
  \item tmake-1.11
  \item qt-2.3.2
  \item qt-embedded-2.3.7
  \item qtopia-free-1.7.0
\end{itemize}
	在编译过程中需要设定以下几个环境变量:
\begin{itemize}\itemsep=-4pt
  \item QTEDIR, 始终指向 qt-embedded-2.3.7 目录
  \item QPEDIR, 始终指向 qtopia-free-1.7.0 目录
  \item QTDIR, 在编译 qt-x11 时指向 qt-2.3.2, 编译 QTE 和 QPE 时指向
		qt-embedded-2.3.7.
  \item TMAKEDIR, 开发Qt环境软件时指向 tmake-1.11(本实验暂未用到)
  \item PATH,首先,在进行交叉编译时需要包含交叉编译器的路径;其次,编译Qt的
		每一阶段都需要包含当前的 \$QTDIR/bin 路径。
  \item LD\_LIBRARY\_PATH, 共享库链接路径,应包含现阶段\$QTDIR/lib路径。
\end{itemize}
\subsection{编译过程}
	下面按顺序对以上涉及到的软件进行编译。
\subsubsection{触摸屏库}
	触摸屏是嵌入式系统典型的人机接口设备。使用方便,成本低廉。。。。。
我们通常所说的``触摸屏''实际上由两部分组成:显示屏和触摸板。这里要编译的主要
是针对触摸板的。但它又不能绝对地跟显示屏无关,因为触摸板的坐标需要跟显示屏
对应。

	通过以下三个步骤完成编译:
\begin{enumerate}\itemsep=-4pt
  \item ./autogen.sh,自动生成configure 命令,供下一步使用
  \item ./configure -{}-host=arm-linux,根据-{}-host选项,生成
		使用交叉编译器的 Makefile。如果再指定-{}-prefix=install\_path, 
可以随后用 make install 将成品安装在 install\_path 目录。
  \item make
\end{enumerate}
	许多从源码编译的软件都需要 configure 配置 Makefile。configure 会自动检查
开发环境是否符合编译条件,并安排合理的编译选项。。。。。。。

	编译时出现一个错误,原因在链接命令的选项 --rpath 要求绝对路径。把相应的
Makefile略作修改(--rpath \`{}cd \$(PLUGIN\_DIR) \&\& pwd\`{}) 重新编译正常
通过,生成若干.so 文件。把它们复制到\$QTEDIR/lib目录。这些共享库在编译QTE时
要用到,系统运行时作为动态链接也需要用到。\footnote{使用-{}-cache-file选项,
config.cache内容为``ac\_cv\_func\_malloc\_0\_nonnull=yes'',该错误不出现}

	运行 make install 时还会生成触摸屏校准和测试程序。初次编译时缺少这一步,
导致目标机 Linux 系统启动后不出现触摸屏校准界面。
\subsubsection{主机Qt环境}
	Qt2是一个相对独立的软件,使用x86编译器编译。即使不做ARM平台的Qt移植,它也
是PC机的重要应用软件。著名的KDE桌面环境采用的就是 Qt 库。这里编译Qt2的目的是
为了用它开发Qt源程序,如,designer 是用于界面设计的,uic 是将 designer 设计
的界面文件转成c语言格式。。。。

	有些PC机可能已经安装了Qt,但通常版本都比较高,对Qt2支持不好。移植QTE2
版本需要预先配置好Qt2。

	Qt2 配置命令:

	./configure -no-xft

	用 ./configure -{}-help 可以列出该软件支持的配置选项。配置Qt2的选项
没有特殊要求。如果不是为主机使用,原则上越简单越好,因为最终只需要生成格式
转换工具而不是用于桌面。

	编译过程中出现一个错误,include/qvaluestack.h -- 57行。从网上查到这是一个
非常典型的问题。原因是 c++ 模板定义中没有默认类成员变量和成员函数规则,必须
用 this-$>$ 显式地指明(不同版本的编译器对语法要求导致)。类似``错误''还有很多,
修改也很容易,只需要加上``this''指针即可。

	将编译生成的 uic 复制到 \$QTEDIR/bin,其他暂时未用到。
\subsubsection{嵌入式Qt环境}
	QTE编译,打补丁。。。。。(为什么要打补丁,补丁的内容)

	配置命令:

	./configure -xplatform linux-arm-g++ -qconfig qpe

	编译命令:

	make sub-src

	错误 include/qwindowsystem\_qws.h:229: error: 'QWSInputMethod' has not been declared。在该文件前面加上两类声明:class QWSInputMethod;
 class QWSGestureMethod;

\vskip 10cm
\subsubsection{Qt桌面环境}
	在 \$QPEDIR 下面设置 ./configure -xplatform linux-arm-g++,并运行make。
注意使用前面生成的 uic,将其路径设在\$PATH 环境的最前面。

	(uuid\footnote{uuid:Universal Unique IDentifier}问题:
	下载e2fsprogs,编译生成 libuuid。或者在链接qpe时去掉-luuid)。
\vskip 6cm
\section{Qt安装}
\subsection{整理编译结果}
	将前面QPE编译后生成的 bin、etc、lib、plugins、i18n、apps、pics目录统一归
到一个目录qpe下面。这些目录各自内容如下:
\begin{itemize}\itemsep=-4pt
  \item bin---可执行程序
  \item etc---QPE配置文件
  \item lib---QPE和QTE库,字体,libqte.so和字体文件从QTE复制过来
  \item i18n---多语言支持
  \item apps---桌面配置文件
  \item pics---图片
  \item plugins---插件,包括各个执行程序的动态链接库
\end{itemize}
\subsection{系统重构}
	重新构造内核,选中Framebuffer支持,Framebuffer中选中Qtopia的GUI、触摸屏
支持。

	重新构造文件系统,将qpe整体迁移到 /usr目录下,为QPE创建一个脚本文件qpe.sh,
内容如下:\\
...\\ ...\\ ...\\

并在启动脚本中增加启动qpe.sh的条目。

	用mkfs.jffs2制作文件系统。

	mkfs.jffs2 -o rootqt.img ......

	重构文件系统不是必须的。为方便起见,可以用原有文件系统通过NFS调试,
逐渐添加和修改Qt配置。

\section{实验结果和分析}
	下载内核和文件系统,烧写Flash,启动。不成功!若干问题及排查按顺序整理如下。
每次手工执行qpe.sh启动QPE:
\begin{enumerate}
  \item 若干库无法链接(libjpeg、libcrypt、libgcc、libdl)\\
	从交叉编译工具目录下复制这些库到 /lib 目录
  \item 提示'could not read calibration /etc/pointercal'\\
	qpe.sh脚本文件中看到,在启动qpe之前有一项 ts\_calibrate 程序,用来校准
	触摸屏。由此得知触摸屏校准文件是 /etc/pointercal,格式未知。随便建立一个
	文件,让系统跳过这一步。
  \item 触摸屏设备打不开\\
	看来是驱动没有编译进内核。查看内核源码 ~drivers/misc/ucb1x00\_ts.c~已编译。
	注释掉脚本中的 MOUSE\_PROTO,改用鼠标协议。
  \item 再次启动QPE,终于出现桌面的第一个画面
\begin{center}
  \includegraphics[width=8cm]{First.png}\end{center}
  \item 换鼠标协议,用 USB 鼠标\\
	export QWS\_MOUSE\_PROTO=USB:/dev/input/mice
	
	成功启动到Linux的Qtopia桌面。
\end{enumerate}
	Qtopia桌面由四个板块构成:应用、游戏、文档、设置。。。。
\section{实验小结}本实验是□□□(以下略去8个字)\vskip 3cm
\section{参考文献}

\end{document}

