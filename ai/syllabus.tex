%%============================================================================
%%
%%       Filename:  syllabus.tex
%%
%%    Description:  creative/innovation course syllabus
%%
%%        Version:  1.0
%%        Created:  04/27/2017 04:25:31 PM
%%       Revision:  none
%%
%%         Author:  Fang Yuan (yfang@nju.edu.cn)
%%   Organization:  nju
%%      Copyright:  Copyright (c) 2017, Fang Yuan
%%
%%          Notes:  
%%                
%%============================================================================
\documentclass{article}
\usepackage[nofonts]{ctex}
\usepackage[left=1.2in,right=1in,top=3.5cm,bottom=1in]{geometry}
\usepackage{tabularx}

\setCJKmainfont[BoldFont={WenQuanYi Micro Hei},
              ItalicFont={STFangsong}]{AR PL SungtiL GB}

%\setmainfont{Times New Roman}
\setsansfont{Droid Sans}
\setmonofont{DejaVu Sans Mono}

\renewcommand\thesection{\chinese{section}.} 
\title{``智能系统中的嵌入式应用''  教学大纲}
\author{大纲起草人:方元}

\begin{document}
\maketitle

\setlength\parindent{1em}

\section{课程信息}

\textbf{(一)基本信息}

课程号:

课程中文名称:智能系统中的嵌入式应用

课程英文名称:Embedded Applications on Intelligent Systems

周学时:2

学分:

先修课程: 程序设计语言

建议教材:自编讲义,自组织教材内容

参考资料:(书籍、文献、网站信息等)

 \begin{enumerate}
    \item 关于嵌入式开发板, 使用官方参考文档
        {\em  Element14 BeagleBone Black System Reference manual}
         和
         {\em AM335x and AMIC110 SitaraTM Processors
         Technical Reference Manual} \\
         Beagleboard 主页   www.beagleboard.org
    \item 关于 Python 语言部分,\\
         {\em Python基础教程(第二版)},
          作者 Magnus Lie Hetland, 曾军崴、谭颖华、司维译,
          人民邮电出版社2010.7\\
          {\em  Programming Python, 2nd Edition}
          By Mark Lutz. O'Reilly, March 2001\\
        Python 主页 www.python.org
    \item 电子实验中心服务器上的教学视频
    \item 网络资源: 嵌入式Linux Wiki, www.elinux.org
 \end{enumerate}

\textbf{(二)内容简介}

   嵌入式系统应用日益广泛。随着性能的不断提升和硬件成本的不断下降,越来越多
   的嵌入式应用结合了高性能的操作系统,导致嵌入式开发的门槛不断降低。
   本课程基于目前较为常见的板卡式计算机,学习用Python 语言实现计算机的控制。
   Python 是一种解释性语言,避免了编译的过程。绝大多数 Linux 的发行版都默认
   安装了 Python语言。Python 语言简单易用,并且是跨平台的,在不同操作系统之间
   保持了很好的源码兼容性,在计算机系统维护、Internet、数据库编程、图像、
   人工智能、机器人等领域有越来越多的应用。本课程以 Linux + Python 为基础
   实现一种寻迹小车、避障、人机对话等等的智能应用系统。

    本课程介绍嵌入式系统中的应用程序开发以及智能化系统的实现。课程内容按
    如下方式组织:
 \begin{enumerate}
    \item 嵌入式计算机系统的构成,嵌入式计算机和通用计算机的异同。
    \item 计算机系统的软件组织方式,计算机软件在计算机系统中的地位和作用
    \item 嵌入式计算机系统的使用和操作方法
	\item Python 语言编程方法
    \item 嵌入式系统 I/O 端口的控制
	\item 使用基本 I/O 端口实现输入/输出功能
	\item 以寻迹小车为模型,介绍智能化系统的实现过程
 \end{enumerate}
        
\textbf{(三)内容简介(英文)}

 (略)


\section{教学目标和学习要求}
通过本课程学习,了解嵌入式应用领域的发展趋势,学习智能系统的初步开发方法。

在硬件方面,要求学生了解嵌入式系统的基本构成和I/O接口的基本知识,
在软件方面,要求学生熟悉嵌入式系统开发的一般过程,了解一些常用的嵌入式系统
开发工具和开发方法,熟悉嵌入式系统的典型应用及产品设计开发的步骤,
初步具备软件与硬件综合测试与调试技能。

\section{其他}
   根据教学需要,补充以下知识点:
  \begin{itemize}
   \item 嵌入式系统的构成,软件开发和维护方法。
  \end{itemize}

\end{document}
