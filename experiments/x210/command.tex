\documentclass[a4paper]{article}

\usepackage[colorlinks=true, 
            linkcolor=black]{hyperref}
\usepackage{graphicx, subfig, multicol, multirow, fancyhdr}
\usepackage[left=0cm,right=0cm,top=2mm,bottom=2mm]{geometry}
\usepackage{fontspec}
\usepackage{fancybox}
\usepackage{colortbl}
\usepackage[table]{xcolor}

\usepackage{tikz}
\usetikzlibrary{positioning,fit,petri}
\usetikzlibrary{decorations}
\usetikzlibrary{plothandlers}
\usetikzlibrary{backgrounds}
\usetikzlibrary{calc}
\usetikzlibrary{matrix}
\usetikzlibrary{shapes}
\usetikzlibrary{arrows}

\makeatletter
\def\hlinewd#1{%
  \noalign{\ifnum0=`}\fi\hrule\@height #1\futurelet%
   \reserved@a\@xhline}
\makeatother
\newcommand\Trule{\hlinewd{1.5pt}}
\newcommand\Brule{\hlinewd{1.5pt}}

\makeatletter
\def\thline{%
  \noalign{\ifnum0=`}\fi
    \penalty\@M
    \futurelet\@let@token\LT@@thline{1.5pt}}
\def\LT@@thline#1{%
    \global\let\@gtempa\@empty
    \gdef\LT@sep{\penalty-\@lowpenalty\vskip-#1}%
  \ifnum0=`{\fi}%
  \multispan\LT@cols
     \unskip\leaders\hrule\@height #1\hfill\cr
  \noalign{\LT@sep}%
  \multispan\LT@cols
     \unskip\leaders\hrule\@height #1\hfill\cr
  \noalign{\penalty\@M}%
  \@gtempa}
\makeatother

\setmainfont{Times New Roman}
\setsansfont{DejaVu Sans}
\setmonofont{DejaVu Sans Mono}
\begin{document}

\definecolor{gray2}{rgb}{0.8,0.8,0.8}

\tt
\begin{center}
\shadowbox{\Huge Unix/Linux Command Reference}
\end{center}

\begin{multicols}{2}

\newcommand{\subtitle}[1]{
\setlength\shadowsize{2pt}
\setlength\fboxrule{1pt}
\shadowbox{\fcolorbox{gray}{yellow!30!white}{\parbox{8cm}{
	\vskip 5pt \large \ \ \color{blue} #1 \vskip 5pt}}}
\ \\
}


\scriptsize
\setlength\fboxsep{0pt}
\subtitle{File Commands}

\shadowbox{
\begin{tabular}{>{\columncolor{gray2}}ll}
   ls  & Directory listing\\
   ls -al& Formatted listing with hidden files\\
   cd dir& Change directory to {\color{red} dir}\\
   cd & Change to home directory\\
   pwd & Show current directory\\
   mkdir dir & Create a directory {\color{red} dir}\\
   rm file & Delete {\color{red} file}\\
   rm -r dir & Delete directory {\color{red} dir}\\
   rm -f file & Force remove {\color{red} file}\\
   rm -rf dir & Force remove directory {\color{red} dir}\\
   cp file1 file2 & Copy {\color{red} file1} to {\color{red} file2}\\
   cp -r dir1 dir2 & Copy {\color{red} dir1} to {\color{red} dir2};create
            {\color{red}
	    dir2} if it doesn't\\
		 & exist\\
   mv file1 file2 & Rename or move {\color{red} file1} to {\color{red} file2}\\
   ln [-s] file link & Create [symbolic] link {\color{red} link} to 
     {\color{red} file}\\
   touch file & Create or update {\color{red} file}\\
   cat $>$file & Places standard input into {\color{red} file}\\
   more file & Output the contents of {\color{red} file}\\
   head file & Output the first 10 lines of {\color{red} file}\\
   tail -f file & Output the contens of {\color{red} file} as it grows,\\
              & starting with the last 10 lines
\end{tabular}
} \ \\ \ \\

\subtitle{Process Management}

\shadowbox{
\begin{tabular}{>{\columncolor{gray2}}ll}
   ps & Display all current active processes\\
   top & Display all running processes\\
   kill pid & Kill process id pid\\
   killall proc & Kill all processes named {\color{red} proc}\\
   bg & List stopped or background jobs; resume a \\
       & stopped job in the background\\
   fg  & Bring the most recent job to the foreground\\
   fg a & Bring job a to the foreground
\end{tabular}
} \ \\ \ \\

\subtitle{File Permissions}

\shadowbox{
\begin{tabular}{>{\columncolor{gray2}}ll}
  chmod octal file & Change the permissions of file to octal,\\
	  & which can be found separately for user,\\
	  & group and world by adding:\\
	  & {\bf 4 -- read(r)}\\
	  & {\bf 2 -- write(w)}\\
	  & {\bf 1 -- execute(x)}\\
	  & Examples:\\
	  & chmod 777 -- read, write, execute for all\\
	  & chmod 755 -- rwx for owner, rx- for group,\\
	  & and world. For more options  man chmod\\
\end{tabular}
} \ \\ \ \\

\subtitle{Searching}

\shadowbox{
\begin{tabular}{>{\columncolor{gray2}}ll}
  grep pattern files & Search for pattern in {\color{red}files}\\
  grep -r pattern dir & Search recursively for pattern in {\color{red}dir}\\
  command | grep pattern & Search for pattern in the output  \\
            & command\\
  locate file & Find instances of {\color{red} file}
\end{tabular}
} \ \\ \ \\

\subtitle{Network}

\shadowbox{
\begin{tabular}{>{\columncolor{gray2}}ll}
  ping host & ping {\color{red}host} and output result\\
  whois domain & get whois information for domain\\
  dig domain & get DNS information for domain\\
  dig -x host & reverse lookup host\\
  wget URL:file & download {\color{red}file} from network\\
  wget -c file & continue a stopped download\\
\end{tabular}
} \ \\ \ \\

\subtitle{System Info}

\shadowbox{
\begin{tabular}{>{\columncolor{gray2}}ll}
  date  & Show the current date and time\\
  cal   & Show this month's calendar\\
  uptime & Show current uptime\\
  w  & Display who is online\\
  whoami & Who you are logged in as\\
  finger user & Display information about user\\
  uname -a & Show kernel information\\
  cat /proc/cpuinfo & CPU information\\
  cat /proc/meminfo & Show memory information\\
  man command & Show the manual for command\\
  df & Show disk usage\\
  du & Show directory space usage\\
  free & Show memory and swap usage\\
  whereis app & Show possible locations of app\\
  which app & Show which app will be run by default\\
\end{tabular}
} \ \\ \ \\

\subtitle{Compression}

\shadowbox{
\begin{tabular}{>{\columncolor{gray2}}ll}
  tar cf file.tar files & Create a tar named {\color{red}file.tar} \\
	   &  containing {\color{red}files}\\
  tar xf file.tar & Extract files from file.tar\\
  tar czf file.tar.gz file & Create a tar with Gzip \\
       & compression\\  
  tar xzf file.tar.gz & Extract a tar using Gzip\\
  tar cjf file.tar.bz2 & Create a tar with Bzip2 \\
        & compression\\
  tar xjf file.tar.bz2 & Extract a tar using Bzip2\\
  gzip file & Compress file and renames it to\\
       & {\color{red}file.gz}\\
  gzip -d file.gz & Decompress {\color{red}file.gz} back to file\\
\end{tabular}
} \ \\ \ \\

\subtitle{Installation}

\shadowbox{
\begin{tabular}{>{\columncolor{gray2}}ll}
  ./configure &\\
  make       & \\
  make install &\\
  rpm -Uvh pkg.rpm & Install a package(RPM)\\
  dpkg -i pkg.deb & Install a package(Debian)\\
  apt-get install & Install package(s) from repository
\end{tabular}
} \ \\ \ \\

\subtitle{Shortcuts}

\shadowbox{
\begin{tabular}{>{\columncolor{gray2}}ll}
 Ctrl+C &  Halt the current command\\
 Ctrl+Z &  Stop the current command, resume with fg in the \\
        & foreground or bg in the background\\
 Ctrl+D &  Logout of current session, similar to exit\\
 Ctrl+W &  Erase one word in the current line\\
 Ctrl+U & Erase the whole line\\
 Ctrl+R & Type to bring up a recent command\\
   !!  &   Repeat the last command\\
  $*$    & \textcolor{red}{USE WITH EXTREME CAUTION}\\
\end{tabular}
} \ \\ \ \\

\end{multicols}

\end{document}

