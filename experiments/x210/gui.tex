\chapter{图形用户接口}
\addthumb{图形用户接口}{\bf 图形}{white}{gray}

\section{实验目的}
\begin{itemize}\itemsep=-3pt
  \item 了解嵌入式系统图形界面的基本编程方法
  \item 学习图形库的制作
\end{itemize}

\section{原理概述}
\subsection{Frame Buffer}
	显示屏的整个显示区域,在系统内会有一段存储空间与之对应.通过改变该存储
空间的内容达到改变显示信息的目的.该存储空间被称为 ~Frame Buffer,~或显存.
显示屏上的每一点都与 ~Frame Buffer~ 里的某一位置对应.所以,解决显示屏的显示
问题,首先需要解决的是 ~Frame Buffer~ 的大小以及屏上的每一像素与 ~Frame
Buffer~ 的映射关系.

	按照显示屏的性能或显示模式区分,显示屏可以以单色或彩色显示.单色用 1 位
来表示(单色并不等于黑与白两种颜色,而是说只能以两种颜色来表示.通常取允许
范围内颜色对比度最大的两种颜色).彩色有 2、4、8、16、24、32等位色.这些色调
代表整个屏幕所有像素的颜色取值范围.如:采用 8 位色/像素的显示模式,显示屏上
能够出现的颜色种类最多只能有 $2^8$ 种.究竟应该采取什么显示模式,首先必须
根据显示屏的性能,然后再由显示的需要来决定.这些因素会影响 ~Frame Buffer~
空间的大小,因为 ~Frame Buffer~ 空间的计算大小是以屏幕的大小和显示模式来
决定的.另一个影响 ~Frame Buffer~ 空间大小的因素是显示屏的单/双屏幕模式.

	单屏模式表示屏幕的显示范围是整个屏幕.这种显示模式只需一个 ~Frame Buffer~
来存储整个屏幕的显示内容,并且只需一个通道来将 ~Frame Buffer~ 内容传输到
显示屏上~(Frame Buffer~ 的内容可能需要被处理后再传输到显示屏).双屏模式则将
整个屏幕划分成两部分.它有别于将两个独立的显示屏组织成一个显示屏.单看其中
一部分,它们的显示方式是与单屏方式一致的,并且两部分同时扫描,工作方式是独
立的.这两部分都各自有 ~Frame Buffer,~且他们的地址无需连续(这里指的是下
半部的 ~Frame Buffer~ 的首地址无需紧跟在上半部的地址末端),并且同时具有独立的
两个通道将 ~Frame Buffer~ 的数据传输到显示屏.

	~Frame Buffer~ 通常就是从内存空间分配所得,并且它是由连续的字节空间组成.
由于屏幕的显示操作通常是从左到右逐点像素扫描、从上到下逐行扫描,直到扫描到
右下角,然后再折返到左上角,而 ~Frame Buffer~ 里的数据则是按地址递增的顺序被
提取,当 ~Frame Buffer~ 里的最后一个字节被提取后,再返回到 ~Frame Buffer~ 的
首地址,所以屏幕同一行上相邻的两像素被映射到 Frame Buffer 里是连续的,某一行的
最末像素与它下一行的首像素反映在 ~Frame Buffer~ 里也是连续的,并且屏幕上最左
上角的像素对应 ~Frame Buffer~ 的第一单元空间,最右下角的像素对应 ~Frame
Buffer~ 的最后一个单元空间.

\subsection{Frame Buffer与色彩}
	计算机反映自然界的颜色是通过 ~RGB~(Red--~Green--~Blue)~值来表示的.如果要在
屏幕某一点显示某种颜色,则必须给出相应的 ~RGB~ 值.~Frame Buffer~ 为屏幕提供
显示的内容,就必须能够从 ~Frame Buffer~ 里得到每一个像素的 ~RGB~ 值.像素的
~RGB~ 值可以直接从 ~Frame Buffer~ 里得到,或是从是调色板间接得到(此时~Frame
Buffer~ 存放的并不是 RGB 值,而是调色板的索引值.通过索引值可以获得调色板的
~RGB 值~).

	Frame Buffer 是由所有像素的 ~RGB~ 值或 ~RGB~ 值的部分位(RGB~ 由红、绿、蓝
各 8 位组成,共 24 位,称为真彩色.由于某些显示屏的数据线有限,只有 16 条
数据线或更少,这时只能取 ~R、~G、~B 部分位与数据线对应)所组成.例如,16 位/像素
模式下,~Frame Buffer~ 里的每个单元为 16 位,每个单元代表一个像素的 RGB 值
(RGB565),如下图.\\

\small
\begin{tabular}{|*{16}{c|}}
\hline
   D15 & D14 & D13 & D12 & D11 & D10 & D9 & D8 &
        D7 & D6 & D5 & D4 & D3 & D2 & D1 & D0\\\hline
  R & R & R & R & R & G & G & G & G & G & G & B & B & B & B & B\\\hline
\end{tabular}
\ \\
\normalsize

	有了以上的分析,就可以用下面的计算公式
$$
	FrameBufferSize=\frac{Width \times Height \times Bit\_per\_Pixel}{8}
$$
计算 ~Frame Buffer~ 的大小(以字节为单位).

\subsection{LCD控制器}
	在 ~Frame Buffer~ 与显示屏之间还需要一个中间件,该中间件负责从 ~Frame
Buffer~ 里提取数据,进行处理,并传输到显示屏上.

	处理器内部集成 LCD 控制器, 将~Frame Buffer~ 里的数据传输到 ~LCDC~ 的内部,
然后经过处理,输出数据到~LCD~ 的输入引脚上.

	本实验系统使用的是32位 LCD,像素分辨率~8000$\times$480.~相关驱动程序在
\$(KERNEL-PATH)/~drivers/~video/~samsung/~s3cfb.c中.

\subsection{Frame Buffer操作}
	Frame Buffer设备是 /dev/fb~(它通常是字符设备 /dev/fb0 的符号链接,该设备
主设备号是29,次设备号是0).了解这个设备的参数可以通过 ~FBIOGET\_FSCREENINFO、
~FBIOGET\_VSCREENINFO~ 命令,如:

\begin{boxedminipage}{.9\textwidth}
\begin{verbatim}
......
struct fb_var_screeninfo vinfo; // #include <linux/fb.h>
......
fd = open ( "/dev/fb", O_RDWR );
......
ioctl( fd, FBIOGET_VSCREENINFO, &vinfo )
\end{verbatim}
\end{boxedminipage}

可以获得显示器色位、分辨率等信息~(vinfo.bits\_per\_pixel、~vinfo.xres、
~vinfo.yres).

获取 ~Frame Buffer~缓冲区首地址的系统调用是

\begin{boxedminipage}{.9\textwidth}
\begin{verbatim}
unsigned char *fbp = 0;
......
fbp = (unsigned char *)mmap(0, screensize,\
        PROT_READ | PROT_WRITE, MAP_SHARED, fd, 0);
\end{verbatim}
\end{boxedminipage}

screensize 是根据显示器信息计算出的缓冲区大小,通过~mmap()~函数获得的缓冲区
首地址.从该首地址开始、以~screensize~为大小的范围即是显示缓冲区的内存映射
地址.如果采用RGB--24位色,在坐标$(x, y)$处画一个红点,可以用下面的方法:

\begin{boxedminipage}{.9\textwidth}
\begin{verbatim}
// draw_point(int x, int y)

    offset = (y * vinfo.xres + x) * vinfo.bits_per_pixel / 8;
    *(unsigned char *)(fbp + offset + 0) = 255;
    *(unsigned char *)(fbp + offset + 1) = 0;
    *(unsigned char *)(fbp + offset + 2) = 0;
    ......
\end{verbatim}
\end{boxedminipage}

	如果是16位色(RGB565),须根据格式要求将 RGB压缩到 16位,再填充对应字节:
\footnote{不同位端(endian)的处理器,移位及高低字节顺序有所不同.}

\begin{boxedminipage}{.9\textwidth}
\begin{verbatim}
// draw_point(int x, int y)

    offset = (y * vinfo.xres + x) * vinfo.bits_per_pixel / 8;
    color = (Red << 11) | ((Green << 5) & 0x07E0) | (Blue & 0x1F);
    *(unsigned char *)(fbp + offset + 0) = color & 0xFF;
    *(unsigned char *)(fbp + offset + 1) = (color >> 8) & 0xFF;
    ......
\end{verbatim}
\end{boxedminipage}

	将显示缓冲区清零,~memset(fbp, 0, screensize),~即可以实现清屏(黑色)操作.

使用完毕应通过~munmap()~ 释放显示缓冲区.
\section{实验内容}
\subsection{实现基本画图功能}
	在 ~Frame Buffer~基础上编写画点、画线的API函数,供应用程序调用,
实现任意曲线的画图功能.

\subsection{合理的软件结构}
    将调用设备驱动的基本~API函数独立地构成一个函数库,为用户程序屏蔽底层硬件
信息,直接提供一些简单的画图调用.函数库可以是独立编译后的``.o''文件或由归档
管理器~ar~生成的库文件,或是将 ``.o''文件链接而成的共享库``.so''.
(例如,静态库文件名为 ~libfoo.a,或共享库文件名~libfoo.so, 编译链接时可用
``--l foo'' 选项)

\section{实验报告要求}
\begin{itemize}\itemsep=-3pt
  \item 总结 ~Frame Buffer~的操作方法;
  \item 探讨软件结构的层次关系;
  \item 思考:如果一帧显示数据的计算量很大,连续图像的刷新、显示将消耗比较多
		的时间,此时如何较好地实现连续画面的动态显示?
\end{itemize}
