\chapter{\tt mplayer移植}
\section{实验目的}
\begin{itemize}\itemsep=-3pt
  \item 掌握 Linux 系统中应用软件移植的过程和方法
  \item 理解软件层次依赖关系
\end{itemize}

\section{软件介绍}

	MPlayer是一款开源的多媒体播放器,支持几乎所有的音频和视频播放, 以GNU
通用公共许可证发布, 可在各主流操作系统使用, 是 Linux 系统中最重要的播放器
之一。MPlayer 中还包含音视频编码工具 mencoder。MPlayer 本身是基于命令行
界面的程序, 不同操作系统、不同发行版可以为它配置不同的图形界面, 使其外观
多姿多彩。MPlayer 本身也可以编译成 GUI 方式。

	MPlayer 除了可以播放一般的磁盘媒体文件外, 还支持CD、VCD、DVD等多种物理
介质和多种网络媒体 (rtp://、rtsp://、http://、mms:// 等)。视频播放时, 它还
支持多种不同格式的外挂字幕。大部分音视频格式
都能通过 ffmpeg 项目(另一个开源项目, 提供音视频编解码库支持) 的libavcodec
函数库原生支持。对于那些没有开源解码器的格式,MPlayer使用二进制的函数库。
它能直接调用Windows的DLL。

\section{编译准备}

	下载 MPlayer 源代码 MPlayer-1.0rc2.tar.bz2, libmad-0.15.1b.tar.gz和
 zlib-1.2.3.tar.bz2, 分别将其解压。
 libmad 是高品质全定点算法的 MPEG 音频解码库。

	准备一个有操作权限的工作目录, 例如 $\sim$/workspace, 作为下面编译结果的
暂存目录。以后的编译过程中,将编译选项 -{}-prefix 设置为该目录。所有编译完成后
再将其内容移至开发板适当位置。缺省的安装路径是 /usr/local, 该路径其一需要root
权限,其二,它是主机系统的一个重要目录, 如果被目标机架构 arm 的代码覆盖, 会影响
主机的正常工作。

	将交叉编译器路径添加到环境变量 ``PATH'' 中。

\section{编译}
\begin{enumerate}
  \item 进入 libmad-0.15.1b 目录,配置编译环境:

	./configure -{}-enable-fpm=arm -{}-host=arm-none-linux-gnueabi
	-{}-disable-debugging -{}-prefix=/home/student/workspace

	选项 ``-{}-host''是编译器前缀。

  \item 编译及安装:  make install

	在编译时会提示错误: cc1: error: unrecognized command line option
	``-fforce-mem''。这是因为 gcc3.4 或更高版本已经将 fforce-mem 选项去除了。
	只需要在 Makefile 中找到该字符串,将其删除即可。

	编译完成后会将静态库 libmad.a 和动态库 madlib.so 安装到
	/home/student/workspace/lib 目录下。如果是动态链接,需要将动态链接库复制到
	目标系统的 /lib 目录下。如果不想用动态链接,可以在上面的编译选项中添加
	一条 ``-{}-disable-shared''。此原则同样适用于下面的 zlib 库。

  \item 进入 zlib-1.2.3 解压目录, 按下面的步骤编译安装:

	export CC=arm-none-linux-gnueabi-gcc

	./configure -{}-prefix=/home/student/workspace

	make install

  \item 进入 MPlayer 解压目录, 进行如下配置:
\begin{verbatim}
  ./configure \
    --cc=arm-none-linux-gnueabi-gcc \
    --target=arm-linux \
    --enable-static \
    --prefix=/home/student/workspace \
    --disable-mp3lib \
    --disable-dvdread \
    --disable-mencoder \
    --disable-live \
    --enable-mad \
    --disable-armv5te \
    --disable-armv6 \
    --enable-libavcodec_a \
    --enable-ossaudio \
    --extra-cflags='-I /home/student/workspace/include' \
    --extra-ldflags='-L /home/student/workspace/lib'
    \end{verbatim}
	上面最后两个选项用到了之前准备的 libmad 和 libz 的头文件及生成的库文件
	路径。配置正确后,可以用 make 命令编译。如果一切正常,便可在当前目录下
	生成可执行文件 mplayer。注意最后链接时用到的库。如果是动态链接,这些库需要
	复制到目标系统的 /lib 目录。
\end{enumerate}

	最后,尝试在目标机上播放一些音视频文件。

\section{扩展功能}
\begin{enumerate}
  \item 尝试编译具有图形用户界面的 MPlayer 播放器
  \item 用 -{}-enable-mencoder 选项编译mencoder, 并利用它进行音视频编码、
	转码。
\end{enumerate}
