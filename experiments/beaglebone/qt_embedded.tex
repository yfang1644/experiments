\chapter{\tt Qt/Embedded移植}
\section{实验目的}
\begin{itemize}\itemsep=-3pt
  \item 了解嵌入式GUI---Qt/E 软件开发平台的构架;
  \item 学习 Qt/E 移植的基本步骤与方法。
\end{itemize}

\section{Qt/E 介绍}
	Qt/Embedded 是跨平台的 c++ 图形用户界面(GUI)工具包,它是著名的 Qt 开发商
~TrollTech~ 发布的面向嵌入式系统的Qt版本。Qt 是目前 ~KDE~ 等项目使用的 ~GUI~
支持库,许多基于 ~Qt~ 的 ~X-Window~ 程序可以非常方便地移植到嵌入式
~Qt/Embedded~ 版本上。自从 ~Qt/Embedded~ 发布以来,有许多嵌入式 ~Linux~开发商
利用 ~Qt/Embedded~ 进行了嵌入式 ~GUI~ 的应用开发。

	Qt/Embedded 注重于能给用户提供精美的图形界面所需的所有元素,而且其开发过程
是基于面向对象的编程思想,并且~Qt/Embedded~支持真正的组件编程。

	TrollTech~公司所发布的面向嵌入式系统的~QT/E~ 版本提供源代码。用户必须针对
自己的嵌入式硬件平台进行裁剪、编译和移植。尽管~Qt/Embedded~ 可以裁剪到630K,
但它对硬件平台具有较高的要求。目前~Qt/Embedded~库主要针对手持式信息终端。

	本实验主要完成~Qt/Embedded~在嵌入式实验平台上的移植。
\subsection{Qt/E软件包结构}
	Qt/E 系统源码一般包括以下几个软件包:
\begin{itemize}\itemsep=-3pt
  \item 触摸屏支持库 tslib.tar.bz2;
  \item Makefile 生成工具 tmake-1.11.tar.gz,它主要由一些脚本程序组成;
  \item 开发平台编译环境库及工具程序 qt-x11-2.3.2.tar.gz;
  \item 目标平台 Qt/Embedded 核心库。用户可到~TrollTech~ 主页上
		\footnote{ftp://ftp.trolltech.com/qt/source}下载Qt/Embedded的某个
		版本的源代码。本实验提供 qt-embedded-2.3.7.tar.gz;
  \item Qt桌面环境 qtopia-free-1.7.0.tar.gz。
\end{itemize}

\section{Qt/E 编译}
\subsection{设置环境}
	为了以下编译过程顺利进行,先将上面的压缩文件全部解压。假设各自被解压到
下面的目录(一般Qt/E~的移植过程也是按这个顺序进行操作的):
\begin{enumerate}\itemsep=-3pt
  \item ~/work/tslib
  \item ~/work/tmake-1.11
  \item ~/work/qt-2.3.2
  \item ~/work/qt-embedded-2.3.7
  \item ~/work/qtopia-1.7.0
\end{enumerate}

	在编译Qt/Embedded时,用户在PC机上应对编译时所需的环境变量进行设置,这些
设置的主要参数包括:
\begin{itemize}\itemsep=-3pt
  \item QTDIR --- Qt解压后的所在的目录
  \item LD\_LIBRARY\_PATH --- Qt共享库存放的目录
  \item QPEDIR --- qtopia解压后的所在的目录
  \item TMAKEPATH --- tmake编译工具的路径
  \item TMAKEDIR-tmake编译工具的目录
  \item PATH --- 交叉编译工具arm-linux-gcc的路径
\end{itemize}

	根据上面的解压目录,环境变量应作如下设置:

\begin{boxedminipage}{.9\textwidth}
\begin{verbatim}
$ export QTDIR=~/work/qt-embedded-2.3.7
$ export QPEDIR=~/work/qtopia-1.7.0
$ export LD_LIBRARY_PATH=$QTDIR/lib:$LD_LIBRARY_PATH
$ export TMAKEDIR=~/work/tmake-1.11
$ export TMAKEPATH=~/work/tmake-1.11/lib/qws/linux-arm-g++
$ export PATH=~/work/tmake-1.11/bin:/usr/local/arm-linux/bin:$PATH
\end{verbatim}
\end{boxedminipage}

\subsection{编译过程}
\subsubsection{编译触摸屏库}
	Qt/Embedded~支持鼠标和键盘的操作,但现在许多嵌入式系统都使用触摸屏作为输入
设备,所以用户必须将触摸屏的相关操作编译成共享库或静态库。

	触摸屏库的编译过程可参考第\ref{ch-ts}章内容。

	将编译完成后的库复制到 qt-embedded-2.3.7目录:

\begin{boxedminipage}{.9\textwidth}
\begin{verbatim}
$ cp -a src/.libs/* ../qt-embedded-2.3.7/lib
$ cp -a plugins/.libs/*.so ../qt-embedded-2.3.7/lib
\end{verbatim}
\end{boxedminipage}

\subsubsection{编译qt-x11工具}
	在~/work/qt-2.3.2 下执行

\begin{boxedminipage}{.9\textwidth}
\begin{verbatim}
$ export QTDIR=~/work/qt-2.3.2
$ export QTEDIR=~/work/qt-embedded-2.3.7
$ export QPEDIR=~/work/qtopia-free-1.7.0
$ export PATH=$QTDIR/bin:$PATH
$ export LD_LIBRARY_PATH=$QTDIR/lib:$LD_LIBRARY_PATH
$ ./configure -no-opengl -no-xft
$ make
$ make -C tools/qvfb
$ mv tools/qvfb/qvfb bin
$ cp bin/uic $QTEDIR/bin
\end{verbatim}
\end{boxedminipage}

	生成的 uic 和 moc 作为 Qt 应用程序的转换工具,qvfb 是 Qt 开发平台的仿真
工具。

\subsubsection{编译 Qt/Embedded}
	先将补丁文件里的文件替换掉源码包里的对应文件(主要是针对目标平台触摸屏代码
和编译器的设置),然后在~/work/qt-embedded-2.3.7目录下执行

\begin{boxedminipage}{.9\textwidth}
\begin{verbatim} 
$ export QTDIR=~/work/qt-embedded-2.3.7
$ cp ~/work/qtopia-free-1.7.0/src/qt/qconfig-qpe.h src/tools
$ ./configure -xplatform linux-arm-g++ -qconfig qpe -depths 16 -no-qvfb
$ make sub-src
\end{verbatim}
\end{boxedminipage}

	这一步完成后,生成目标平台的 Qt 核心库 libqte.so* 等等。

\subsubsection{编译 Qtopia}
	在 ~/work/qtopia-free-1.7.0/src 下面执行

\begin{boxedminipage}{.9\textwidth}
\begin{verbatim} 
$ ./configure -platform linux-arm-g++
$ make
\end{verbatim}
\end{boxedminipage}

	编译完成后会产生 apps、bin、doc、etc、help、include、plugins等目录及
目录下的文件。至此,编译过程基本结束。

\subsection{Qt/Embedded的安装}
	准备将待安装的文件放在一个独立的目录下。新建一个目录~/work/qpe,将
~qtopia-free-1.7.0/src 下面的~apps、~bin、~etc、~plugins、~i18n、~lib、~pics~
这些目录连同下面的子目录和文件复制到该目录下,同时将 ~qt-embedded-2.3.7/lib~
下面的库连同字体目录也复制到 ~qpe/lib~下(注意保持原来的目录结构)。由于字体
文件比较大,可适当删除一些不常用的字体库,保留 ~*.qpf~ 文件和 ~fontdir~ 文件。
另外,还要将触摸屏配置文件 ~/work/tslib/etc/ts.conf 复制到 ~etc 目录。

	将整个 qpe 目录复制到目标系统文件系统的 ~/usr~ 目录下,再为 ~qpe~建立一个
启动脚本 ~(/usr~/bin~/qpe.sh):

\begin{boxedminipage}{.9\textwidth}
\begin{verbatim} 
$ export QTDIR=/usr/qpe
$ export QPEDIR=/usr/qpe
$ export LANG=zh_CN
$ export LD_LIBRARY_PATH=/usr/qpe/lib:$LD_LIBRARY_PATH
$ export QT_TSLIBDIR=/usr/qpe/lib
$ export TSLIB_CONFFILE=/usr/qpe/etc/ts.conf
$ export TSLIB_PLUGINDIR=/usr/qpe/lib
$ export KDEDIR=/usr/qpe
$ /usr/bin/ts_calibrate
$ /usr/qpe/bin/qpe &> /dev/null
\end{verbatim}
\end{boxedminipage}

	将Qt/E 系统与~BusyBox~ 结合,按第{} \ref{ch-fs}
章的方法重新制作文件系统映像; 根据需要, 修改启动脚本 inittab(或rc)的启动
执行步骤。

\subsection{Qt-4.8版本编译}
目前的Qt最新版本是 5.5\footnote{2015年7月}。新版本支持更多的特性、更好的
人机交互体验,例如支持触摸屏的滑动和多点触控等等。但编译相当耗时。

实验室提供Qt-4.8.4的 opensource 版本源代码。

与低版本编译顺序类似,应先编译触摸屏库,将其安装在指定目录,并在Qt解压目录下面
的 mkspecs/qws/linux-arm-g++/qmake.conf 文件中添加如下几行
\begin{verbatim}
QMAKE_INCDIR    = ~/work/build/include
QMAKE_LIBDIR    = ~/work/build/lib
QMAKE_LIBS      = -lpng -lz -lts
\end{verbatim}

$\sim$work/build 目录为 libpng, libz 和 libts 的交叉编译安装目录。配置 Qt 编译
环境如下命令:

\begin{verbatim}
 ./configure \
    -prefix ~/work/build \
    -opensource \
    -confirm-license \
    -release -shared \
    -embedded arm \
    -xplatform qws/linux-arm-g++ \
    -depths 16,24,32 \
    -fast \
    -optimized-qmake \
    -no-pch \
    -no-largefile \
    -qt-sql-sqlite \
    -system-zlib -system-libtiff -system-libpng -system-libjpeg \
    -qt-freetype \
    -no-qt3support \
    -no-mmx -no-sse -no-sse2 \
    -no-3dnow \
    -no-openssl \
    -no-opengl \
    -webkit \
    -no-phonon \
    -no-nis \
    -no-cups \
    -no-glib \
    -no-xcursor -no-xfixes -no-xrandr -no-xrender \
    -no-separate-debug-info \
    -make libs -make examples -make tools -make docs \
    -qt-mouse-tslib -qt-mouse-pc -qt-mouse-linuxtp
\end{verbatim}

编译完成后,将 $\sim$/work/build 目录平移到目标文件系统,仿照低版本的
按照方式设置环境变量。

\section{实验要求}
完成一个 Qt/Embedded 系统的编译和安装。

\newpage
\begin{center} ****************** \end{center}
\tt [附] 编译过程中的一些错误及修正

	由于编译器版本及源代码规范性等方面的原因,对源码软件编译过程中经常会
碰到一些编译错误。对这些编译错误,最好能根据编译器给出的错误提示,找到出错
的地方,有针对性地加以修正。下面是在编译Qt/E 过程中可能出现的一些错误及解决
办法:
\begin{enumerate}\itemsep=-3pt
  \item {\bf qt-2.3.2}: include/qvaluestack.h:57: 错误......\\
		修改 include/qvaluestack.h 第 57 行,将\\
		remove( this-$>$fromLast() ); 改为\\
		this-$>$remove( this-$>$fromLast() );
  \item {\bf qt-embedded-2.3.7}: include/qwindowsystem\_qws.h:229: error:
		'QWSInputMethod' has not been declared\\
		在 include/qwindowsystem\_qws.h 里加上类声明:\\
		class   QWSInputMethod;\\
		class   QWSGestureMethod;
  \item {\bf qt-embedded-2.3.7}: *** [allmoc.o] 错误 1\\
		向前追溯出错位置,在 include/qsortedlist.h 中,将第51行改为:\\
		$\sim$QSortedList() \{ this-$>$clear(); \}\\
		同样性质的``错误''还有很多,取决于编译器版本。这里不再一一列举。
  \item {\bf qtopia-free-1.7.0}: Makefile:10: *** 遗漏分隔符 。 停止。\\
		修改 src/3rdparty/libraries/libavcodec/Makefile,删除 ~10、~14、~18~
		行的 --e
  \item {\bf qtopia-free-1.7.0}: libraries/qtopia/backend/event.cpp:404:
		error: ISO C++ says that these are ambiguous,......\\
		c++对操作符``$<$=''理解有歧义。可将局部变量 i 声明为 int。
  \item {\bf qtopia-free-1.7.0}: libraries/qtopia/qdawg.cpp:243: error:
		extra qualification 'QDawgPrivate::......\\
		去掉类定义中的本类声明。
  \item {\bf qtopia-free-1.7.0}: libavformat/img.c:723: error: static
		declaration of 'pgm\_iformat' follows non-static declaration\\
		函数或变量属性声明冲突。
  \item {\bf qtopia-free-1.7.0}: 对'\_\_cxa\_guard\_release'未定义的引用\\
		出现在链接阶段。到编译器路径下找到该函数或变量所属的库(libstdc++.so),
		在编译 ~qt-embedded-2.3.7~ 时加上库的链接。
\end{enumerate}
\rm
