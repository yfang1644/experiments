\documentclass[nofonts]{ctexart}
\usepackage{indentfirst}                % 每章、节首段缩进
\usepackage[CJKbookmarks,               % 中文书签
            colorlinks=true,            % 允许彩色链接
            linkcolor=magenta]{hyperref}% 超链接包
\usepackage{graphicx, subfigure, multicol, multirow, fancyhdr}
\usepackage{listings}
\usepackage[a4paper,left=2cm,right=2cm,top=3cm,bottom=3cm]{geometry}
\usepackage{fontspec}
\usepackage{longtable}
\usepackage{tikz}
\usepackage{amsfonts}

\setCJKfamilyfont{caiyun}{STCaiyun}
\setCJKfamilyfont{songti}{AR PL SungtiL GB}
\setCJKfamilyfont{fangsong}{STFangsong}
\setCJKfamilyfont{heiti}{WenQuanYi Zen Hei}
\setCJKfamilyfont{kaiti}{AR PL UKai CN}
\setCJKfamilyfont{xinwei}{STXinwei}
\setCJKfamilyfont{lishu}{LiSu}
\setCJKmainfont[BoldFont={WenQuanYi Zen Hei}, ItalicFont={STFangsong}]{AR PL SungtiL GB}
\setCJKsansfont{WenQuanYi Zen Hei}
\setCJKmonofont{AR PL UKai CN}

\newcommand*{\song}{\CJKfamily{songti}}   % 宋体
\newcommand*{\fs}{\CJKfamily{fangsong}}   % 仿宋
\newcommand*{\hei}{\CJKfamily{heiti}}     % 黑体
\newcommand*{\kai}{\CJKfamily{kaiti}}     % 楷书
\newcommand*{\wei}{\CJKfamily{xinwei}}    % 新魏
\newcommand*{\lishu}{\CJKfamily{lishu}}   % 隶书
\newcommand*{\cy}{\CJKfamily{caiyun}}     % 彩云

\CTEXsetup[name={第,章}]{section}
\CTEXsetup[format={\raggedright}]{section}
\CTEXsetup[nameformat={\Large \hei}]{section}
\CTEXsetup[titleformat={\Large \hei}]{section}
%\CTEXsetup[aftername={\vskip 3ex}]{section}
\CTEXsetup[format+={\Large}]{section}

\setlength{\parindent}{2em}             % 缩进2个字母M的宽度
\title{Raspberry Pi2 开发基础}
\date{2015年6月29日}

\begin{document}
\lstset{language=C++}
\lstset{extendedchars=false}
%\lstset{keywordstyle=\fontfamily{phv}\fontseries{b}\fontshape{n}\selectfont}
\lstset{numbers=none}

\maketitle

\tableofcontents

\newpage
\section{准备工作}
\begin{enumerate}
  \item 安装交叉编译器 arm-linux-gcc
  \item 格式化 TF 卡。

  将 TF 卡分区。第一个分区用 VFAT 格式化,作为boot分区;第二个分区用 inode 文件
系统格式化(ext2fs,ext3fs,ext4fs, reiserfs,btrfs,jffs2 等等,要求内核支持),
准备作为 Linux 系统的根文件系统。

\begin{verbatim}
  mkfs.vfat /dev/mmcblk0p1
  mkfs.ext4 /dev/mmcblk0p2
\end{verbatim}

把 bootloader 的文件和内核镜像 zImage 复制到第一个分区。bootloader 的
配置文件 config.txt 中 ``kernel=zImage''用来指导 bootloader 加载内核,
cmdline.txt 用来向内核传递启动参数。
\end{enumerate}

\section{内核编译}
内核编译的步骤
\begin{enumerate}
  \item 配置好编译器,设定可执行程序路径(arm-linux-gcc的路径加入 PATH);
  \item 设定待编译内核的处理器架构环境变量 export ARCH=arm;
  \item 下载内核源码,解压,进入源码主目录;
  \item arch/arm/configs 下面有很多平台预设的缺省配置,如果可用,
  执行``make foo\_defconfig'',可以把对应的配置文件 foo\_defconfig 复制到
  源码主目录 .config。否则,继续下一步;
  \item ``make menuconfig'',配置内核功能;
  \item ``make''编译内核,如无错误,会在 arch/arm/boot 目录下生成 zImage。
\end{enumerate}


\section{根文件系统制作}
\subsection{编译 busybox}
  busybox 集成了 Linux 的基本操作命令,从基本的目录操作文件操作(cd, ls, rm,
  mkdir, rmdir, more...)到常用的网络服务(ifconfig, ftp/fptd, telnet/telnetd)。

\begin{enumerate}
  \item 下载 busybox, 解压,进入busybox
  \item make menuconfig, 配置选项,根据需要精简。
  \item make, make install, 缺省安装后在 \_install 目录下生成 bin, sbin,
    usr/bin, usr/sbin。
\end{enumerate}

\subsection{完善目录结构}
  在 \_install 目录下创建空目录 etc, lib, dev, mnt, proc, root, home。

  etc 目录存放系统配置文件:
\begin{enumerate}
  \item 创建并编辑 etc/inittab,它是系统1号进程``init''的脚本

  \begin{verbatim}
  # /etc/inittab
  ::sysinit:/etc/init.d/rcS
  ::askfirst:-/bin/sh
  ::once:/usr/sbin/telnetd -l /bin/login
  tty0::respawn:/sbin/getty 115200 tty0
  ::ctrlaltdel:/sbin/reboot
  ::shutdown:/bin/umount -a -r
  \end{verbatim}

  它指导 init 做下面的事情:
  \begin{itemize}
    \item 执行 /etc/init.d/rcS (此文件下面建立)
	\item 控制台使用 /bin/sh
	\item 启动 telnet 服务(telnet daemon),调用登录命令 /bin/login
	\item 在终端 tty0 上出现登录界面
	\item CTRL+ALT+DEL组合键执行 reboot 命令
	\item 关机时卸载所有文件系统
  \end{itemize}
  \item 创建并编辑文件 etc/rc

  \begin{verbatim}
  #!/bin/sh
  hostname RaspBerryPi2
  mount -t proc proc /proc
  cat /etc/motd
  ifconfig eth0 192.168.1.121
  mount -t sysfs sysfs /sys
  mkdir /dev/pts
  mount -t devpts devpts /dev/pts
  \end{verbatim}

  设置可执行权限属性``chmod +x rc'',做一个符合链接到 etc/init.d/rcS

  ln -s rc init.d/rcS (先要建 init.d 目录)

	此文件被 init 通过 inittab 脚本调用,完成下面的工作:
  \begin{itemize}
    \item 设置主机名
	\item 挂载虚拟文件系统 proc, sysfs devpts。/dev/pts 用于 telnet 服务。
    \item 配置网络接口 (ifconfig)
	\item 其他的初始化任务 (例如 打印 motd--Message Of ToDay)
  \end{itemize}
  \item 为了打印 motd ,需要在 etc目录下创建文件 motd
  \begin{verbatim}
  ====================
    Welcome to
      Raspberry PI2
  ====================
  \end{verbatim}
\end{enumerate}

   为了实现远程登录,需要给系统添加帐号。管理帐号的配置文件 passwd 和 group
   也要放在 etc 目录下。passwd 形式如下:
   \begin{verbatim}
   root::0:0:root:/root:/bin/sh
   user::1000:1000:user:/home/user:/bin/sh
   \end{verbatim}

   group 形式如下:
   \begin{verbatim}
   root::0:
   \end{verbatim}

   lib 目录存放库文件:
\begin{enumerate}
  \item 把交叉编译器路径下的一些共享库(动态链接库)及其符号链接复制到 lib 或者
     usr/lib 目录。重要的库有下面几个

     ld-2.13.so, libc-2.13.so, libm-2.13.so

     libgcc\_s.so.1, libstdc++.so.6.0.16

  \begin{verbatim}
  ln –s ld-2.13.so ld-linux.so.3
  ln –s libc-2.13.so libc.so.6
  ln –s libm-2.13.so libm.so.6

  ln -s libgcc_s.so.1 libgcc_s.so
  ln -s libstdc++.so.6.0.16 libstdc++.so.6
  ln -s libstdc++.so.6.0.16 libstdc++.so
  \end{verbatim}
\end{enumerate}

   如果内核配置了支持 devtmpfs (Device Drivers--Generic Driver Options),
     且bootloader 传递给内核的参数中包含了 devtmpfs.mount=1,则只需要留一个空的
	 dev 目录。否则,需要在 dev 目录下创建几个设备文件:

  \begin{verbatim}
  mknod console c 5 1
  mknod null c 1 3
  mknod zero c 1 5
  \end{verbatim}

\subsection{制作根文件系统}
  将以上完成的目录复制到 TF 卡第二个分区的根目录下面。在 cmdline.txt 中用
  ``root=/dev/mmcblk0p2 rw rootfstype=ext4'' 告诉内核在哪里加载根文件系统。


\section{通过frame buffer 操作图形设备}
frame buffer 是一种内存映射机制,把物理设备的内存映射到用户空间。通常用于
图形设备。下面是一个例子:

\begin{lstlisting}[caption={[清单]fbinit.c},label=fb]{}
#include <stdio.h>
#include <linux/fb.h>
#include <strings.h>
#include <sys/ioctl.h>
#include <sys/types.h>
#include <sys/stat.h>
#include <fcntl.h>
#include <sys/mman.h>

int fd;
void *fbp;
struct fb_var_screeninfo vinfo;
struct fb_fix_screeninfo finfo;
void fb_init(void)
{
    unsigned fbsize;
    unsigned int x, y, color;

    fd = open("/dev/fb0", O_RDWR);
    ioctl(fd, FBIOGET_FSCREENINFO, &finfo);
    ioctl(fd, FBIOGET_VSCREENINFO, &vinfo);
    printf("x = %d, y = %d\n", vinfo.xres, vinfo.yres);
    printf("bits = %d\n", vinfo.bits_per_pixel);

    fbsize = vinfo.yres*finfo.line_length;
    fbp =(char *)mmap(NULL, fbsize, PROT_READ|PROT_WRITE, MAP_SHARED, fd, 0);

    if(vinfo.bits_per_pixel == 32) {
        x = 200, y = 200;
        color = 0x00FF0080;
       	*(unsigned int *)(fbp + y*finfo.line_length + x) = color;
    } else if(vinfo.bits_per_pixel == 24) {
        ...
    }
}
\end{lstlisting}

frame buffer的设备文件是 /dev/fb0。ioctl 函数用来获取显示器的规格(分辨率,
每像素的比特数等等)。32位的显示是R8,G8,B8,A8,24位的显示是 R8,G8,B8,
16位显示通常是 R5,G6,B5,也可能是R5,G5,B5,A1,更早的时候还有8位的显示模式。

mmap() 返回的指针就是显示缓存的线性地址空间。如需要在屏幕坐标(x,y)处画一个
点,只需要在这个内存空间对应的地址上填入相应的颜色配比就可以了。

\section{Linux 设备驱动}
\subsection{设备驱动程序的结构}
\begin{lstlisting}[caption={[清单]gpio.c},label=gpio]{}
#include <linux/kernel.h>
#include <linux/module.h>
#include <linux/fs.h>
#include <asm/io.h>
#include <linux/init.h>
#include <asm/uaccess.h>

#define    GPFSEL0   0x3f200000   /* function select register */
#define    GPSET0    0x3f20001c   /* bit set */
#define    GPCLR0    0x3f200028   /* bit clear */
#define    GPLEV0    0x3f200034   /* level holder register */

volatile int *gpio_fsel;
volatile int *gpio_clr;
volatile int *gpio_set;
volatile int *gpio_level;

ssize_t gpio_read(struct file *fp, char *buff, size_t n, loff_t *f){
    int val, i;
    val = *gpio_level;
    i = copy_to_user(buff, &val, 4);
    return 4;
}

ssize_t gpio_write(struct file *fp, const char *buff, size_t n, loff_t *f){
    int val, i;
    i = copy_from_user(&val, buff, 4);
    printk("write %d\n", val);
    val = val & 4;
    if (val)
        *gpio_set = val;
    else
        *gpio_clr = (~val & 4);
    return 4;
}

int gpio_open(struct inode *node, struct file *fp){
    int var = *gpio_fsel;
    var &= ~(7<<6);
    var |= (1<<6);             /* set GPIO2 as output (001) */
    printk("GPFSEL = %x\n", var);
    *gpio_fsel = var;
    return 0;
}

int gpio_close(struct inode *node, struct file *fp){
    return 0;
}

struct file_operations gpio ={
    read: gpio_read,
    write: gpio_write,
    open: gpio_open,
    release: gpio_close,
};

int init_module(){
    int n;
    n = register_chrdev(123, "gpiogpio", &gpio);
    if(n==0) {
        printk("device registered\n");

        gpio_fsel  = ioremap(GPFSEL0, 4);
        gpio_set   = ioremap(GPSET0, 4);
        gpio_clr   = ioremap(GPCLR0, 4);
        gpio_level = ioremap(GPLEV0, 4);
        return 0;
    } else
        return -1;
}

void cleanup_module(){
    unregister_chrdev(123, "gpiogpio");
    iounmap(gpio_fsel);
    iounmap(gpio_set);
    iounmap(gpio_clr);
    iounmap(gpio_level);
    printk("device removed\n");
}
\end{lstlisting}

驱动程序(又称内核模块)和普通用户空间运行的程序有以下不同:
\begin{enumerate}
  \item 没有主函数 main(),自己不能独立运行,需要靠OS调用。
	用户操作层面上有两个函数入口:init\_module(),安装驱动(``insmod gpio.ko'')
	时被调用; cleanup\_module(), 卸载驱动(``rmmod gpio'')时被调用。
  \item 运行方式不同。通过内核注册相应功能(简单的设备用 register\_chrdev()
	函数注册),等待用户程序调用。
  \item 运行空间不同。内核模块运行在内核空间,拥有最高特权级。用户程序
  运行在用户空间,没有对系统硬件的控制权,必须通过内核模块实现功能。
  用户空间和内核空间的指针也不能混用,必须通过 copy\_to\_user()和
  copy\_from\_user() 复制存储单元的内容。
  \item 不能使用用户程序的函数(比如printf),include 的 head 文件必须是内核
  源码路径下的文件。
\end{enumerate}

\subsection{编译设备驱动}
编译设备驱动必须要在安装了对应版本的内核源码 head files 环境中才能进行。

\lstset{language=make}
\begin{lstlisting}{}
# Makefile (!not makefile)
ifneq ($(KERNELRELEASE),)
obj-m       := gpio.o
else
KDIR        := /home/user/raspberrypi2/linux-3.18.y
PWD         := $(shell pwd)
default:
        $(MAKE) -C $(KDIR) SUBDIRS=$(PWD) modules
endif
\end{lstlisting}

KDIR 是内核源码路径,PWD是待编译的驱动程序源码路径(一般是当前路径,用pwd命令
获取)。编译后生成 gpio.ko,这就是设备驱动程序的二进制代码。

上例中,驱动程序注册了一个字符设备(register\_chrdev),(主)设备号123,
设备名称是 ``gpiogpio'',注册的功能在 file\_operations 结构包含的函数组中。
其中 read, write, open, release 由用户程序的系统调用函数 read, write, open,
close 调用。

\subsection{设备的使用}

设备文件是用户程序和驱动程序之间的桥梁。创建设备文件用命令:

mknod device c 123 0

上面的命令创建了一个文件 device , 这个文件是字符设备``c'',主设备号123,
次设备号0。一般设备文件被集中放在 /dev 目录下。

\begin{lstlisting}[caption={[清单]control.c},label=control]{}
#include <sys/types.h>
#include <sys/stat.h>
#include <fcntl.h>
#include <unistd.h>
#include <stdio.h>

int main()
{
    int fd;
    int val;

    fd = open("/dev/device", O_RDWR);

    for(;;) {
        val = 4;
        write(fd, &val, 4);
        sleep(1);
        val = 0;
        write(fd, &val, 4);
        sleep(1);
    }
    return 0;
}
\end{lstlisting}

编写驱动程序的原则是,只提供策略,不强加功能。只解决能做什么的问题,而不是
去具体实施怎么做。

\end{document}

